% TIKZ preamble and setup for external runs
\usetikzlibrary{calc}
\tikzset{Bfield/.style={decorate, double}}


%Code for externalizing the tikz image generations
\iffalse
\tikzfeynmanset{compat=1.1.0}
\immediate\write18{mkdir -p images}  %% Create `pgf-img` directory
\usetikzlibrary{external}  
          %% Load the `external` library
\immediate\write18{mkdir -p images}  %% Create `pgf-img` directory
\tikzexternalize[                     %% Activate externalization
  prefix=images/,                    %% Avoid cluttering the directory
 system call={                       %% Use lualatex in system call
    pdflatex \tikzexternalcheckshellescape -halt-on-error -shell-escape -interaction=batchmode -jobname="\image" "\texsource" || rm "\image.pdf"
  },
]
\fi

%Style definitions
\tikzfeynmanset{
  fermion2/.style={
    /tikz/postaction={
      /tikz/decoration={
        markings,
        mark=at position 0.7 with {
          \arrow{>[length=6pt, width=7pt]};
        },
      },
      /tikz/decorate=true,
    },
  },
}

% Put the baseline at the center of the picture, for every
% picture. The shift makes it closer to being centered in the center
% of the line (for intance with equations) rather than the center
% sitting on the baseline itself (which in equations makes things look
% not-centered against operators)
\tikzset{every picture/.style={baseline={([yshift=-.7ex]current bounding box.center)}}}

% Diagram definitions

%% Nice blob definition consistent with what James was already using,
%% but handling the coordinate duplication internally.
\newcommand{\blob}[2]{\vertex[dot,scale=2] (#1) at (#2){};\vertex[dot,scale=1.5,hgrey0] at (#2){};}
%% Use the same syntax, so that it is easy to convert between blobby
%% cuts and normal diagrams
\newcommand{\ver}[2]{\coordinate (#1) at (#2){};}



\newcommand{\KTDBContrib}[4]{ {
    \begin{tikzpicture}
      \pgfmathsetmacro{\cutlength}{0.8}
\begin{feynman}
\vertex (a1) at (-1.5,1){#4};
\vertex (a2) at (1.5,-1){#2};
\vertex (a3) at (1.5,1){#3};
\vertex (a4) at (-1.5,-1){#1};
\vertex (mid1) at (0,-.5);
\vertex (mid2) at (0,.5);
\vertex (mid3) at (1,.5) ;
\vertex (mid4) at (1,-.5) ;
\vertex (mid5) at (-1,.5);
\vertex (mid6) at (-1,-.5) ;
% Cut reference coords at the center of each line
%LH box midpoints
\vertex (m1) at ($(mid5)!0.5!(mid6)$);
\vertex (m2) at ($(mid1)!0.5!(mid6)$);
\vertex (m3) at ($(mid5)!0.5!(mid2)$);
\vertex (clo) at ($(m2)!0.5!(m3)$) {};
\vertex (c1) at ($(clo)+(90:\cutlength)$);
\vertex (c2) at ($(clo)+(180:\cutlength)$);
\vertex (c3) at ($(clo)+(270:\cutlength)$);
%RH box midpoints
\vertex (m4) at ($(mid3)!0.5!(mid4)$);
\vertex (m5) at ($(mid3)!0.5!(mid2)$);
\vertex (m6) at ($(mid4)!0.5!(mid1)$);
\vertex (cro) at ($(m5)!0.5!(m6)$) {};
\vertex (c4) at ($(cro)+(90:\cutlength)$);
\vertex (c5) at ($(cro)+(0:\cutlength)$);
\vertex (c6) at ($(cro)+(270:\cutlength)$);


\diagram{
(a4) --[ultra thick,](mid6),
(a3) --[ultra thick,](mid3),
(a2) --[ultra thick,](mid4),
(a1) --[ultra thick,](mid5),
(mid1) --[ultra thick,ucp-color](mid2),
(mid3) --[ultra thick,](mid2),
(mid1) --[ultra thick,](mid6),
(mid1) --[ultra thick,](mid4),
(mid5) --[ultra thick,](mid2),
(mid5) --[ultra thick,](mid6),
(mid3) --[ultra thick,](mid4),
(clo)--[ultra thick,dashed,cut-color](c1),
(clo)--[ultra thick,dashed,cut-color](c2),
(clo)--[ultra thick,dashed,cut-color](c3),
(cro)--[ultra thick,dashed,cut-color](c4),
(cro)--[ultra thick,dashed,cut-color](c5),
(cro)--[ultra thick,dashed,cut-color](c6)
};
\end{feynman}
\end{tikzpicture}
}
}


%%% Jacobi 1 %%%

\newcommand{\JacobiOneDoubleBox}{ {
\begin{tikzpicture}
\begin{feynman}
\vertex (a1) at (-1.5,1){4};
\vertex (a2) at (1.5,-1){2};
\vertex (a3) at (1.5,1){3};
\vertex (a4) at (-1.5,-1){1};
\vertex (mid1) at (0,-.5);
\vertex (mid2) at (0,.5);
\vertex (mid3) at (1,.5) ;
\vertex (mid4) at (1,-.5) ;
\vertex (mid5) at (-1,.5);
\vertex (mid6) at (-1,-.5) ;
\diagram{
(a4) --[ultra thick,](mid6),
(a3) --[ultra thick,](mid3),
(a2) --[ultra thick,](mid4),
(a1) --[ultra thick,](mid5),
(mid1) --[ultra thick,](mid2),
(mid3) --[ultra thick,](mid2),
(mid1) --[ultra thick,](mid6),
(mid1) --[ultra thick,](mid4),
(mid5) --[ultra thick,jacobi-color](mid2),
(mid5) --[ultra thick,](mid6),
(mid3) --[ultra thick,](mid4),
};
\end{feynman}
\end{tikzpicture}
}
}

\newcommand{\JacobiOneCrossedBox}{ {
\begin{tikzpicture}
\begin{feynman}
\vertex (a1) at (-1.5,1){4};
\vertex (a2) at (1.5,-1){2};
\vertex (a3) at (1.5,1){3};
\vertex (a4) at (-1.5,-1){1};
\vertex (mid1) at (0,-.5);
\vertex (mid2) at (0,.5);
\vertex (mid3) at (1,.5);
\vertex (mid4) at (1,-.5);
\vertex (mid5) at (-1,.5);
\vertex (mid6) at (-1,-.5);
\vertex (mid7) at (-.5,0) {};
\diagram{
(a4) --[ultra thick,](mid6),
(a3) --[ultra thick,](mid3),
(a2) --[ultra thick,](mid4),
(a1) --[ultra thick,](mid5),
(mid1) --[ultra thick,](mid6),
(mid3) --[ultra thick,](mid2),
(mid1) --[ultra thick,](mid4),
(mid5) --[ultra thick,](mid7),
(mid7) --[ultra thick,](mid1),
(mid2) --[ultra thick,](mid6),
(mid5) --[ultra thick,jacobi-color](mid2),
(mid3) --[ultra thick,](mid4),
};
\end{feynman}
\end{tikzpicture}
}
}

\newcommand{\JacobiOnePentaTriangle}{ {
\begin{tikzpicture}
\begin{feynman}
\vertex (a1) at (-1.5,1){4};
\vertex (a2) at (.5,-1){2};
\vertex (a3) at (.5,1){3};
\vertex (a4) at (-1.5,-1){1};
\vertex (mid1) at (-.5,-.5);
\vertex (mid2) at (-1,0);
\vertex (mid3) at (0,.5) ;
\vertex (mid4) at (0,-.5) ;
\vertex (mid5) at (-1,.5);
\vertex (mid6) at (-1,-.5) ;
\diagram{
(a4) --[ultra thick,](mid6),
(a3) --[ultra thick,](mid3),
(a2) --[ultra thick,](mid4),
(a1) --[ultra thick,](mid5),
(mid1) --[ultra thick,](mid2),
(mid6) --[ultra thick,](mid2),
(mid1) --[ultra thick,](mid6),
(mid1) --[ultra thick,](mid4),
(mid5) --[ultra thick,](mid3),
(mid5) --[ultra thick,jacobi-color](mid2),
(mid3) --[ultra thick,](mid4),
};
\end{feynman}
\end{tikzpicture}
}
}

%%% Jacobi 1 end %%%

%%% Jacobi 2 start %%%

\newcommand{\JacobiTwoCrossedBoxOne}{ {
\begin{tikzpicture}
\begin{feynman}
\vertex (a1) at (-1.5,1){2};
\vertex (a2) at (1.5,-1){4};
\vertex (a3) at (1.5,1){3};
\vertex (a4) at (-1.5,-1){1};
\vertex (mid1) at (0,-.5);
\vertex (mid2) at (0,.5);
\vertex (mid3) at (1,.5);
\vertex (mid4) at (1,-.5);
\vertex (mid5) at (-1,.5);
\vertex (mid6) at (-1,-.5);
\vertex (mid7) at (-.5,0) {};
\diagram{
(a4) --[ultra thick,](mid6),
(a3) --[ultra thick,](mid3),
(a2) --[ultra thick,](mid4),
(a1) --[ultra thick,](mid5),
(mid1) --[ultra thick,](mid6),
(mid3) --[ultra thick,](mid2),
(mid1) --[ultra thick,jacobi-color](mid4),
(mid5) --[ultra thick,](mid7),
(mid7) --[ultra thick,](mid1),
(mid2) --[ultra thick,](mid6),
(mid5) --[ultra thick,](mid2),
(mid3) --[ultra thick,](mid4),
};
\end{feynman}
\end{tikzpicture}
}
}

\newcommand{\JacobiTwoCrossedBoxTwo}{ {
\begin{tikzpicture}
\begin{feynman}
\vertex (a1) at (-1.5,1){3};
\vertex (a2) at (1.5,-1){4};
\vertex (a3) at (1.5,1){1};
\vertex (a4) at (-1.5,-1){2};
\vertex (mid1) at (0,-.5);
\vertex (mid2) at (0,.5);
\vertex (mid3) at (1,.5);
\vertex (mid4) at (1,-.5);
\vertex (mid5) at (-1,.5);
\vertex (mid6) at (-1,-.5);
\vertex (mid7) at (-.5,0) {};
\diagram{
(a4) --[ultra thick,](mid6),
(a3) --[ultra thick,](mid3),
(a2) --[ultra thick,](mid4),
(a1) --[ultra thick,](mid5),
(mid1) --[ultra thick,](mid6),
(mid3) --[ultra thick,](mid2),
(mid1) --[ultra thick,jacobi-color](mid4),
(mid5) --[ultra thick,](mid7),
(mid7) --[ultra thick,](mid1),
(mid2) --[ultra thick,](mid6),
(mid5) --[ultra thick,](mid2),
(mid3) --[ultra thick,](mid4),
};
\end{feynman}
\end{tikzpicture}
}
}

\newcommand{\JacobiTwoCrossedBoxThree}{ {
\begin{tikzpicture}
\begin{feynman}
\vertex (a1) at (-1.5,1){1};
\vertex (a2) at (1.5,-1){4};
\vertex (a3) at (1.5,1){2};
\vertex (a4) at (-1.5,-1){3};
\vertex (mid1) at (0,-.5);
\vertex (mid2) at (0,.5);
\vertex (mid3) at (1,.5);
\vertex (mid4) at (1,-.5);
\vertex (mid5) at (-1,.5);
\vertex (mid6) at (-1,-.5);
\vertex (mid7) at (-.5,0) {};
\diagram{
(a4) --[ultra thick,](mid6),
(a3) --[ultra thick,](mid3),
(a2) --[ultra thick,](mid4),
(a1) --[ultra thick,](mid5),
(mid1) --[ultra thick,](mid6),
(mid3) --[ultra thick,](mid2),
(mid1) --[ultra thick,jacobi-color](mid4),
(mid5) --[ultra thick,](mid7),
(mid7) --[ultra thick,](mid1),
(mid2) --[ultra thick,](mid6),
(mid5) --[ultra thick,](mid2),
(mid3) --[ultra thick,](mid4),
};
\end{feynman}
\end{tikzpicture}
}
}


%%% Jacobi 2 end %%%

%%% Graph sym start %%%

\newcommand{\CrossedBoxGraphSym}{ {
\begin{tikzpicture}
\begin{feynman}
\vertex (a1) at (-1.5,1){};
\vertex (a2) at (1.5,-1){};
\vertex (a3) at (1.5,1){};
\vertex (a4) at (-1.5,-1){};
\vertex (mid1) at (0,-.5);
\vertex (mid2) at (0,.5);
\vertex (mid3) at (1,.5);
\vertex (mid4) at (1,-.5);
\vertex (mid5) at (-1,.5);
\vertex (mid6) at (-1,-.5);
\vertex (mid7) at (-.5,0) {};
\vertex (x1) at (-1.5,0);
\vertex (x2) at (1.5,0);
\diagram{
(a4) --[ultra thick,](mid6),
(a3) --[ultra thick,](mid3),
(a2) --[ultra thick,](mid4),
(a1) --[ultra thick,](mid5),
(mid1) --[ultra thick,](mid6),
(mid3) --[ultra thick,](mid2),
(mid1) --[ultra thick,](mid4),
(mid5) --[ultra thick,](mid7),
(mid7) --[ultra thick,](mid1),
(mid2) --[ultra thick,](mid6),
(mid5) --[ultra thick,](mid2),
(mid3) --[ultra thick,](mid4),
(x1) --[ultra thick,dashed,refl-color](x2),
};
\end{feynman}
\end{tikzpicture}
}
}

%%% Graph sym end %%%

%%% Unitarity cuts begin %%%

\newcommand{\PhysicalCutOne}[4]{ {
\begin{tikzpicture}[baseline=6]
\begin{feynman}
\vertex (a1) at (-1, 1) {#1};
\vertex (a2) at (1, 1) {#2};
\vertex [dot,scale=2](mid1) at (0.5,0.5){};
\vertex [dot,scale=1.5,hgrey0](mid2) at (0.5,0.5){};
\vertex [dot,scale=2](mid3) at (-0.5,0.5){};
\vertex [dot,scale=1.5,hgrey0](mid4) at (-0.5,0.5){};
\vertex [dot,scale=2](mid5) at (0,0){};
\vertex [dot,scale=1.5,hgrey0](mid6) at (0,0){};
\vertex (a3) at (-.5, -.5) {#3};
\vertex (a4) at (.5, -.5) {#4};
\diagram{
(mid3) --[ ultra thick](a1),
(mid1) --[ ultra thick](a2),
(mid1) --[ ultra thick,out=120,in=60,min distance=0.1cm](mid3),
(mid1) --[ ultra thick](mid3),

(mid1) --[ ultra thick](mid5),
(mid3) --[ ultra thick](mid5),

(mid5) --[ ultra thick](a4),
(mid5) --[ ultra thick,](a3)
};
\end{feynman}
\end{tikzpicture}
}
}

\newcommand{\PhysicalCutTwo}[4]{ {
\begin{tikzpicture}
\begin{feynman}
\vertex (a1) at (-.6, -0.6) {#1};
\vertex (a2) at (-.6, 0.6) {#2};
\vertex [dot,scale=2](mid1) at (0,0){};
\vertex [dot,scale=1.5,hgrey0](mid2) at (0,0){};
\vertex [dot,scale=2](mid3) at (.8,0){};
\vertex [dot,scale=1.5,hgrey0](mid4) at (.8,0){};
\vertex [dot,scale=2](mid5) at (1.6,0){};
\vertex [dot,scale=1.5,hgrey0](mid6) at (1.6,0){};
\vertex (a3) at (2.2, .6) {#3};
\vertex (a4) at (2.2, -.6) {#4};
\diagram{
(mid1) --[ ultra thick,](a1),
(mid1) --[ ultra thick,](a2),
(mid1) --[ ultra thick,out=60,in=120,min distance=0.4cm](mid3),
(mid1) --[ ultra thick,out=-60,in=-120,min distance=0.4cm](mid3),
(mid3) --[ ultra thick,out=60,in=120,min distance=0.4cm](mid5),
(mid3) --[ ultra thick,out=-60,in=-120,min distance=0.4cm](mid5),
(mid5) --[ ultra thick](a4),
(mid5) --[ ultra thick,](a3)
};
\end{feynman}
\end{tikzpicture}
}
}

\newcommand{\MaxCut}{ {
\begin{tikzpicture}
\begin{feynman}
\vertex (a1) at (-1.2,0.8){};
\vertex (a2) at (1.2,-0.8){};
\vertex (a3) at (1.2,0.8){};
\vertex (a4) at (-1.2,-0.8){};
\vertex [dot,scale=2] (mid1) at (0,-.4){};
\vertex [dot,scale=1.5,hgrey0](mid1x) at (0,-.4){};
\vertex [dot,scale=2] (mid2) at (0,.4){};
\vertex [dot,scale=1.5,hgrey0] (mid2x) at (0,.4){};
\vertex [dot,scale=2] (mid3) at (0.8,.4){};
\vertex [dot,scale=1.5,hgrey0] (mid3x) at (0.8,.4){};
\vertex [dot,scale=2] (mid4) at (0.8,-.4){};
\vertex [dot,scale=1.5,hgrey0] (mid4x) at (0.8,-.4){};
\vertex [dot,scale=2] (mid5) at (-0.8,.4){};
\vertex [dot,scale=1.5,hgrey0] (mid5x) at (-0.8,.4){};
\vertex [dot,scale=2] (mid6) at (-0.8,-.4){};
\vertex [dot,scale=1.5,hgrey0] (mid6x) at (-0.8,-.4){};
\diagram{
(a4) --[ultra thick,](mid6),
(a3) --[ultra thick,](mid3),
(a2) --[ultra thick,](mid4),
(a1) --[ultra thick,](mid5),
(mid1) --[ultra thick,](mid2),
(mid3) --[ultra thick,](mid2),
(mid1) --[ultra thick,](mid6),
(mid1) --[ultra thick,](mid4),
(mid5) --[ultra thick,](mid2),
(mid5) --[ultra thick,](mid6),
(mid3) --[ultra thick,](mid4),
};
\end{feynman}
\end{tikzpicture}
}
}

\newcommand{\SubtleCut}{ {
\begin{tikzpicture}
\begin{feynman}
\vertex (a1) at (-.8, -0.6) {};
\vertex (a2) at (-.8, 0.6) {};
\vertex (a3) at (-1, 0) {};
\vertex [dot, scale=2](mid1) at (0,0){};
\vertex [dot, scale=1.5,hgrey0](mid2) at (0,0){};
\vertex [dot, scale=2](mid3) at (1,0){};
\vertex [dot, scale=1.5,hgrey0](mid4) at (1,0){};
\vertex (a4) at (1.8, 0) {};
\diagram{
(mid1) --[ultra thick,](a1),
(mid1) --[ultra thick,](a2),
(mid1) --[ultra thick,](a3),
(mid1) --[ultra thick,](mid3),
(mid1) --[ultra thick,out=60,in=120,min distance=0.4cm](mid3),
(mid1) --[ultra thick,out=-60,in=-120,min distance=0.4cm](mid3),
(mid3) --[ultra thick](a4),
};
\end{feynman}
\end{tikzpicture}
}
}

\newcommand{\LMCut}{ {
\begin{tikzpicture}
\begin{feynman}
\vertex (a1) at (-.8, -0.6) {};
\vertex (a2) at (-.8, 0.6) {};
\vertex [dot, scale=2](mid1) at (0,0){};
\vertex [dot, scale=1.5,hgrey0](mid2) at (0,0){};
\vertex [dot, scale=2](mid3) at (1,0){};
\vertex [dot, scale=1.5,hgrey0](mid4) at (1,0){};
\vertex (a3) at ($(mid3)+(0.8, -0.6)$) {};
\vertex (a4) at ($(mid3)+(.8, 0.6)$) {};
\diagram{
(mid1) --[ultra thick,](a1),
(mid1) --[ultra thick,](a2),
(mid1) --[ultra thick,](mid3),
(mid1) --[ultra thick,out=60,in=120,min distance=0.4cm](mid3),
(mid1) --[ultra thick,out=-60,in=-120,min distance=0.4cm](mid3),
(mid3) --[ultra thick](a4),
(mid3) --[ultra thick,](a3),
};
\end{feynman}
\end{tikzpicture}
}
}

\newcommand{\KissingTriangles}{
    \begin{tikzpicture}
      \begin{feynman}
        \pgfmathsetmacro{\ri}{0.8};
        \pgfmathsetmacro{\ro}{0.6};
        \pgfmathsetmacro{\defl}{25};
        \blob{kiss}{0,0};
        \blob{e1}{\defl:\ri};
        \blob{e2}{-\defl:\ri};
        \blob{e3}{180+\defl:\ri};
        \blob{e4}{180-\defl:\ri};
        \vertex (i1) at ($(e1) + (\defl:\ro)$);
        \vertex (i2) at ($(e2) + (-\defl:\ro)$);
        \vertex (i3) at ($(e3) + (180+\defl:\ro)$);
        \vertex (i4) at ($(e4) + (180-\defl:\ro)$);
        \diagram{
          {[edges={ultra thick}] (e1)--(e2)--(kiss)--(e3)--(e4)--(kiss)--(e1),
            (e1)-- (i1),
            (e2)-- (i2),
            (e3)-- (i3),
            (e4)-- (i4)
          };
        };
    \end{feynman}
  \end{tikzpicture}
}
%%% Unitarity cuts end %%%

\newcommand{\simplebox}{ {
\begin{tikzpicture}
\begin{feynman}
\vertex (a1) at (-1,1){2};
\vertex (a2) at (1,-1){4};
\vertex (a3) at (1,1){3};
\vertex (a4) at (-1,-1){1};
\vertex (mid3) at (.35,.35);
\vertex (mid4) at (.35,-.35);
\vertex (mid5) at (-.35,.35);
\vertex (mid6) at (-.35,-.35);
\diagram{(mid4) --[ultra thick,nhpRed](mid3),
(mid5) --[ultra thick,nhpRed](mid6),
(mid3) --[ultra thick,nhpRed](mid5),
(mid6) --[ultra thick,nhpRed](mid4),
(mid6) --[ultra thick,fermion2](a4),
(mid3) --[ultra thick,fermion2](a3),
(mid4) --[ultra thick,fermion2](a2),
(mid5) --[ultra thick,fermion2](a1),
};
\vertex [dot, scale=1.6](mid3a) at (.35,.35){};
\vertex [dot, scale=1.1,hgrey0](mid3b) at (.35,.35){};
\vertex [dot, scale=1.6](mid4a) at (-.35,.35){};
\vertex [dot, scale=1.1,hgrey0](mid4b) at (-.35,.35){};
\vertex [dot, scale=1.6](mid5a) at (.35,-.35){};
\vertex [dot, scale=1.1,hgrey0](mid5b) at (.35,-.35){};
\vertex [dot, scale=1.6](mid6a) at (-.35,-.35){};
\vertex [dot, scale=1.1,hgrey0](mid6b) at (-.35,-.35){};
\end{feynman}
\end{tikzpicture}
}
}

\newcommand{\cubic}[7]{ {
\begin{tikzpicture}
\begin{feynman}
\vertex [dot, scale=2.6](mid1) at (0,0){};
\vertex [dot, scale=2.6](mid2) at (0,0){};
\vertex [dot, scale=2,#1](mid3) at (0,0){};
\vertex (a1) at (0,1){3};
\vertex (a2) at (.85,-.55){2};
\vertex (a3) at (-.85,-.55){1};
\diagram{
(mid1) --[ultra thick,#2,#3](a3),
(mid1) --[ultra thick,#4,#5](a2),
(a1) --[ultra thick,#6,#7](mid1),
};
\end{feynman}
\end{tikzpicture}
}
}

\newcommand{\AcubicB}[9]{ {
\begin{tikzpicture}
\begin{feynman}
\vertex (a1) at (0,0){};
\vertex (a2) at (-1.55,.85){2};
\vertex (a3) at (-1.55,-.85){1};
\vertex [dot, scale=2.6](mid1) at (-.85,0){};
\vertex [dot, scale=2.6](mid2) at (-.85,0){};
\vertex [dot, scale=2,#1](mid3) at (-.85,0){};
\vertex (a4) at (1.55,.85){3};
\vertex (a5) at (1.55,-.85){4};
\vertex [dot, scale=2.6](mid4) at (.85,0){};
\vertex [dot, scale=2.6](mid5) at (.85,0){};
\vertex [dot, scale=2,#1](mid6) at (.85,0){};
\diagram{
(mid1) --[ultra thick,#2,#3](a3),
(mid1) --[ultra thick,#4,#5](a2),
(mid4) --[ultra thick,#8,#9](a4),
(mid4) --[ultra thick,#8,#9](a5),
(mid1) --[ultra thick,#6,#7](a1),
(a1) --[ultra thick,#6,#7](mid4),
};
\end{feynman}
\end{tikzpicture}
}
}

\newcommand{\Acubic}[7]{ {
\begin{tikzpicture}
\begin{feynman}
\vertex (a1) at (0,1){3};
\vertex (a2) at (.85,-.55){2};
\vertex (a3) at (-.85,-.55){1};
\vertex [dot, scale=2.6](mid1) at (0,0){};
\vertex [dot, scale=2.6](mid2) at (0,0){};
\vertex [dot, scale=2,#1](mid3) at (0,0){};
\diagram{
(a3) --[ultra thick,#2,#3](mid1),
(a2) --[ultra thick,#4,#5](mid1),
(mid1) --[ultra thick,#6,#7](a1),
};
\end{feynman}
\end{tikzpicture}
}
}


\newcommand{\dBox}[7]{ {
\begin{tikzpicture}
\begin{feynman}
\vertex (a1) at (-1.5,1){2};
\vertex (a2) at (1.5,-1){4};
\vertex (a3) at (1.5,1){3};
\vertex (a4) at (-1.5,-1){1};
\vertex (mid1) at (0,-.5);
\vertex (mid2) at (0,.5);
\vertex (mid3) at (1,.5) ;
\vertex (mid4) at (1,-.5) ;
\vertex (mid5) at (-1,.5);
\vertex (mid6) at (-1,-.5) ;
\diagram{
(a4) --[ultra thick](mid6),
(a3) --[ultra thick,](mid3),
(a2) --[ultra thick,](mid4),
(a1) --[ultra thick,](mid5),
(mid1) --[ultra thick,#1](mid2),
(mid3) --[ultra thick,#2](mid2),
(mid1) --[ultra thick,#3](mid6),
(mid1) --[ultra thick,#4](mid4),
(mid5) --[ultra thick,#5](mid2),
(mid5) --[ultra thick,#6](mid6),
(mid3) --[ultra thick,#7](mid4),
};
\end{feynman}
\end{tikzpicture}
}
}


\newcommand{\dBoxR}[7]{ {
\begin{tikzpicture}
\begin{feynman}
\vertex (a1) at (-1.5,1){2};
\vertex (a2) at (1.5,-1){4};
\vertex (a3) at (1.5,1){3};
\vertex (a4) at (-1.5,-1){1};
\vertex (mid1) at (0,-.5);
\vertex (mid2) at (0,.5);
\vertex (mid3) at (1,.5) ;
\vertex (mid4) at (1,-.5) ;
\vertex (mid5) at (-1,.5);
\vertex (mid6) at (-1,-.5) ;
\diagram{
(a4) --[ultra thick,](mid6),
(a3) --[ultra thick,hred](mid3),
(a2) --[ultra thick,hred](mid4),
(a1) --[ultra thick,](mid5),
(mid1) --[ultra thick,#1](mid2),
(mid3) --[ultra thick,#2](mid2),
(mid1) --[ultra thick,#3](mid6),
(mid1) --[ultra thick,#4](mid4),
(mid5) --[ultra thick,#5](mid2),
(mid5) --[ultra thick,#6](mid6),
(mid3) --[ultra thick,#7](mid4),
};
\end{feynman}
\end{tikzpicture}
}
}

\newcommand{\dBoxL}[7]{ {
\begin{tikzpicture}
\begin{feynman}
\vertex (a1) at (-1.5,1){2};
\vertex (a2) at (1.5,-1){4};
\vertex (a3) at (1.5,1){3};
\vertex (a4) at (-1.5,-1){1};
\vertex (mid1) at (0,-.5);
\vertex (mid2) at (0,.5);
\vertex (mid3) at (1,.5) ;
\vertex (mid4) at (1,-.5) ;
\vertex (mid5) at (-1,.5);
\vertex (mid6) at (-1,-.5) ;
\diagram{
(a4) --[ultra thick,hred](mid6),
(a3) --[ultra thick,](mid3),
(a2) --[ultra thick,](mid4),
(a1) --[ultra thick,hred](mid5),
(mid1) --[ultra thick,#1](mid2),
(mid3) --[ultra thick,#2](mid2),
(mid1) --[ultra thick,#3](mid6),
(mid1) --[ultra thick,#4](mid4),
(mid5) --[ultra thick,#5](mid2),
(mid5) --[ultra thick,#6](mid6),
(mid3) --[ultra thick,#7](mid4),
};
\end{feynman}
\end{tikzpicture}
}
}



%%% Local Variables:
%%% mode: latex
%%% TeX-master: "2loopNLSM"
%%% End:
