\documentclass[11pt,letter]{article}
\usepackage{jheppub}
\usepackage{amsmath,amssymb}
\usepackage[dvipsnames]{xcolor}
%\usepackage{hyperref}
\usepackage{xspace}
\usepackage{ifdraft}
\usepackage{epstopdf}
\usepackage{slashed}
\usepackage{diagbox}
\usepackage{colortbl}
\usepackage{bbold}
\usepackage{tabularx}
\usepackage{graphicx}
\usepackage{stackengine}
\def\RA{\rlap{\scalebox{1.6}{$\leftrightarrow$}}}
\def\DA{\smash{\bclap{\scalebox{1.6}{$\downarrow$}}}}
\def\mystrut{\rule[-2ex]{0ex}{6ex}}

\usepackage{tikz-feynman} 
\usepackage{xcolor}
\hypersetup{
    colorlinks,
    linkcolor={nhpRed},
    citecolor={nhpBlue},
    urlcolor={nhpBlue}
}

\definecolor{nhpRed}{RGB}{161,0,0}
\definecolor{nhp4}{RGB}{203, 4, 31}
\definecolor{nhp3}{RGB}{244,99,30}
\definecolor{nhp2}{RGB}{255,159,0}
\definecolor{nhp1}{RGB}{48,152,152}
\definecolor{nhpBlue}{RGB}{0,100,144}
\definecolor{cutred}{RGB}{219,56,49}
\definecolor{hgreen}{RGB}{25,176,146}
\definecolor{hgreen1}{RGB}{175,230,175}
\definecolor{hblue}{RGB}{52,152,219}
\definecolor{hbluedark}{RGB}{36, 106, 160}
\definecolor{hblue1}{RGB}{255,255,166}
\definecolor{hred}{RGB}{216,83,117}
\definecolor{hreddark}{RGB}{151, 58, 81}
\definecolor{hred1}{RGB}{255,155,155}
\definecolor{cutred}{RGB}{219,56,49}
\definecolor{hgrey4}{RGB}{75,75,75}
\definecolor{hgrey5}{RGB}{50,50,50}
\definecolor{hgrey3}{RGB}{100,100,100}
\definecolor{hgrey}{RGB}{125,125,125}
\definecolor{hgrey2}{RGB}{125,125,125}
\definecolor{hgrey1}{RGB}{150,150,150}
\definecolor{hgrey0}{RGB}{175,175,175}
\definecolor{darkgreen}{RGB}{59,126,108}

\tikzset{graviton/.style={decorate, decoration={snake, amplitude=.4mm, segment length=1.5mm, pre length=.5mm, post length=.5mm}, double}}
\tikzfeynmanset{compat=1.1.0}
\immediate\write18{mkdir -p images}  %% Create `pgf-img` directory
\usetikzlibrary{external}  
          %% Load the `external` library
\immediate\write18{mkdir -p images}  %% Create `pgf-img` directory
\tikzexternalize[                     %% Activate externalization
  prefix=images/,                    %% Avoid cluttering the directory
 system call={                       %% Use lualatex in system call
    pdflatex \tikzexternalcheckshellescape -halt-on-error -shell-escape -interaction=batchmode -jobname="\image" "\texsource" || rm "\image.pdf"
  },
]
\tikzfeynmanset{
  fermion2/.style={
    /tikz/postaction={
      /tikz/decoration={
        markings,
        mark=at position 0.7 with {
          \arrow{>[length=6pt, width=7pt]};
        },
      },
      /tikz/decorate=true,
    },
  },
}


\definecolor{jaxoblue}{HTML}{0086FF}

\newcommand{\cP}{\rm{cov.}\pi}

\newcommand{\ie}{i.e.~}
\newcommand{\eg}{e.g.~}
\usepackage{xspace}
\newcommand{\Poincare}{Poincar\'e\xspace}

\newcommand{\cC}{g} % -- Crossing Count symbol.


\newcommand{\nkmc}[1][k]{N$^{#1}$MC\xspace}

\def\taua{{{\rm t}}}
\def\bartaua{{{\bar {\rm t}}}}



\newcommand{\simplebox}{ {
\begin{tikzpicture}[baseline=(current  bounding  box.center)]
\begin{feynman}
\vertex (a1) at (-1,1){2};
\vertex (a2) at (1,-1){4};
\vertex (a3) at (1,1){3};
\vertex (a4) at (-1,-1){1};
\vertex (mid3) at (.35,.35);
\vertex (mid4) at (.35,-.35);
\vertex (mid5) at (-.35,.35);
\vertex (mid6) at (-.35,-.35);
\diagram{(mid4) --[ultra thick,nhpRed,quarter right](mid3),
(mid5) --[ultra thick,nhpRed,quarter right](mid6),
(mid3) --[ultra thick,nhpRed,quarter right](mid5),
(mid6) --[ultra thick,nhpRed,quarter right](mid4),
(mid6) --[ultra thick,fermion2](a4),
(mid3) --[ultra thick,fermion2](a3),
(mid4) --[ultra thick,fermion2](a2),
(mid5) --[ultra thick,fermion2](a1),
};
\end{feynman}
\end{tikzpicture}
}
}

\newcommand{\cubic}[7]{ {
\begin{tikzpicture}[baseline=(current  bounding  box.center)]
\begin{feynman}
\vertex [dot, scale=2.6](mid1) at (0,0){};
\vertex [dot, scale=2.6](mid2) at (0,0){};
\vertex [dot, scale=2,#1](mid3) at (0,0){};
\vertex (a1) at (0,1){3};
\vertex (a2) at (.85,-.55){2};
\vertex (a3) at (-.85,-.55){1};
\diagram{
(mid1) --[ultra thick,#2,#3](a3),
(mid1) --[ultra thick,#4,#5](a2),
(a1) --[ultra thick,#6,#7](mid1),
};
\end{feynman}
\end{tikzpicture}
}
}

\newcommand{\Acubic}[7]{ {
\begin{tikzpicture}[baseline=(current  bounding  box.center)]
\begin{feynman}
\vertex (a1) at (0,1){3};
\vertex (a2) at (.85,-.55){2};
\vertex (a3) at (-.85,-.55){1};
\vertex [dot, scale=2.6](mid1) at (0,0){};
\vertex [dot, scale=2.6](mid2) at (0,0){};
\vertex [dot, scale=2,#1](mid3) at (0,0){};
\diagram{
(a3) --[ultra thick,#2,#3](mid1),
(a2) --[ultra thick,#4,#5](mid1),
(mid1) --[ultra thick,#6,#7](a1),
};
\end{feynman}
\end{tikzpicture}
}
}


\newcommand{\dBox}[7]{ {
\begin{tikzpicture}[baseline=(current  bounding  box.center)]
\begin{feynman}
\vertex (a1) at (-1.5,1){2};
\vertex (a2) at (1.5,-1){4};
\vertex (a3) at (1.5,1){3};
\vertex (a4) at (-1.5,-1){1};
\vertex (mid1) at (0,-.5);
\vertex (mid2) at (0,.5);
\vertex (mid3) at (1,.5) ;
\vertex (mid4) at (1,-.5) ;
\vertex (mid5) at (-1,.5);
\vertex (mid6) at (-1,-.5) ;
\diagram{
(a4) --[ultra thick,](mid6),
(a3) --[ultra thick,](mid3),
(a2) --[ultra thick,](mid4),
(a1) --[ultra thick,](mid5),
(mid1) --[ultra thick,#1](mid2),
(mid3) --[ultra thick,#2](mid2),
(mid1) --[ultra thick,#3](mid6),
(mid1) --[ultra thick,#4](mid4),
(mid5) --[ultra thick,#5](mid2),
(mid5) --[ultra thick,#6](mid6),
(mid3) --[ultra thick,#7](mid4),
};
\end{feynman}
\end{tikzpicture}
}
}


\newcommand{\dBoxR}[7]{ {
\begin{tikzpicture}[baseline=(current  bounding  box.center)]
\begin{feynman}
\vertex (a1) at (-1.5,1){2};
\vertex (a2) at (1.5,-1){4};
\vertex (a3) at (1.5,1){3};
\vertex (a4) at (-1.5,-1){1};
\vertex (mid1) at (0,-.5);
\vertex (mid2) at (0,.5);
\vertex (mid3) at (1,.5) ;
\vertex (mid4) at (1,-.5) ;
\vertex (mid5) at (-1,.5);
\vertex (mid6) at (-1,-.5) ;
\diagram{
(a4) --[ultra thick,](mid6),
(a3) --[ultra thick,hred](mid3),
(a2) --[ultra thick,hred](mid4),
(a1) --[ultra thick,](mid5),
(mid1) --[ultra thick,#1](mid2),
(mid3) --[ultra thick,#2](mid2),
(mid1) --[ultra thick,#3](mid6),
(mid1) --[ultra thick,#4](mid4),
(mid5) --[ultra thick,#5](mid2),
(mid5) --[ultra thick,#6](mid6),
(mid3) --[ultra thick,#7](mid4),
};
\end{feynman}
\end{tikzpicture}
}
}

\newcommand{\dBoxL}[7]{ {
\begin{tikzpicture}[baseline=(current  bounding  box.center)]
\begin{feynman}
\vertex (a1) at (-1.5,1){2};
\vertex (a2) at (1.5,-1){4};
\vertex (a3) at (1.5,1){3};
\vertex (a4) at (-1.5,-1){1};
\vertex (mid1) at (0,-.5);
\vertex (mid2) at (0,.5);
\vertex (mid3) at (1,.5) ;
\vertex (mid4) at (1,-.5) ;
\vertex (mid5) at (-1,.5);
\vertex (mid6) at (-1,-.5) ;
\diagram{
(a4) --[ultra thick,hred](mid6),
(a3) --[ultra thick,](mid3),
(a2) --[ultra thick,](mid4),
(a1) --[ultra thick,hred](mid5),
(mid1) --[ultra thick,#1](mid2),
(mid3) --[ultra thick,#2](mid2),
(mid1) --[ultra thick,#3](mid6),
(mid1) --[ultra thick,#4](mid4),
(mid5) --[ultra thick,#5](mid2),
(mid5) --[ultra thick,#6](mid6),
(mid3) --[ultra thick,#7](mid4),
};
\end{feynman}
\end{tikzpicture}
}
}




\DeclareMathOperator{\cut}{\mathcal{C}}

 \def\draftnote#1{{\color{red}\it #1}} 
\iffalse \def\draftnote#1{{\color{red}\it}} \fi

\def\sect#1{section~\ref{#1}}
\def\Fig#1{fig.~{\ref{#1}}}
\def\Fig#1{Fig.~{\ref{#1}}}
\def\Figs#1#2{figs.~{\ref{#1}} and {\ref{#2}}}
\def\Figs#1#2{Figs.~{\ref{#1}} and {\ref{#2}}}

\def\tab#1{table~{\ref{#1}}}
\def\Tab#1{Table~{\ref{#1}}}
\def\tabs#1#2{tables~{\ref{#1}} and {\ref{#2}}}
\def\Tabs#1#2{TablesnFigs.~{\ref{#1}} and {\ref{#2}}}

\def\spa#1.#2{\left\langle#1\,#2\right\rangle}
\def\spb#1.#2{\left[#1\,#2\right]}
\def\spash#1.#2{\spa{\smash{#1}}.{\smash{#2}}}
\def\spbsh#1.#2{\spb{\smash{#1}}.{\smash{#2}}}
\def\sand#1.#2.#3{%
\left\langle\smash{#1}{\vphantom1}^{-}\right|{#2}%
\left|\smash{#3}{\vphantom1}^{-}\right\rangle}
\def\sandpp#1.#2.#3{%
\left\langle\smash{#1}{\vphantom1}^{+}\right|{#2}%
\left|\smash{#3}{\vphantom1}^{+}\right\rangle}
\def\sandpm#1.#2.#3{%
\left\langle\smash{#1}{\vphantom1}^{+}\right|{#2}%
\left|\smash{#3}{\vphantom1}^{-}\right\rangle}
\def\sandmp#1.#2.#3{%
\left\langle\smash{#1}{\vphantom1}^{-}\right|{#2}%
\left|\smash{#3}{\vphantom1}^{+}\right\rangle}
\def\Shift#1#2{{[#1,#2\rangle}}
\def\twoloop{{2 \mbox{-} \rm loop}}

\def\tree{{\rm tree}}
\def\pol{\varepsilon}
\def\Tr{\, {\rm Tr}}
\def\tr{\, {\rm tr}}
\def\eps{\varepsilon}
\def\e{\varepsilon}
\def\ep{\varepsilon}
\def\SYM{MSYM}
\def\nn{\nonumber}
\def\sec#1{section~\ref{#1}}
\def\eqn#1{eq.~(\ref{#1})}
\def\Eqn#1{Equation~(\ref{#1})}
\def\eqns#1#2{eqs.~(\ref{#1}) and~(\ref{#2})}
\def\Eqns#1#2{Eqs.~(\ref{#1}) and~(\ref{#2})}
\def\Figref#1{Fig.~\ref{#1}}
\def\secref#1{section~\ref{#1}}
\def\Neqfour{{{\cal N}=4}}
\def\NeqFour{{{\cal N}=4}}
\def\Neqeight{{{\cal N}=8}}
\def\NeqEight{{{\cal N}=8}}
\def\NeqOne{{{\cal N}=1}}
\def\Fact{{\cal F}}
\def\f{\widetilde f}
\def\be{\begin{equation}}
\def\ee{\end{equation}}
\def\bea{\begin{eqnarray}}
\def\eea{\end{eqnarray}}
\def\ba{\begin{eqnarray}}
\def\ea{\end{eqnarray}}
\def\Ksl{{\s K}}
\def\ksl{\s{k}}
\def\Perm{{\cal P}}
\def\M{{\cal M}}
\def\ve{\varepsilon}
\def\tlambda{{\tilde\lambda}}
\def\MHVbar{$\overline{\hbox{MHV}}$}
\def\NMHVbar{$\overline{\hbox{NMHV}}$}
\def\P{{\rm P}}
\def\NHP{{\rm NHP}}
\def\mud{\lambda}
\def\bowtie{{\rm bow\mbox{-}tie}}
\newcommand{\cred}{\bf \color{red}}
 \newcommand{\cblue}{\color{blue}}
 \definecolor{MattOrange}{rgb}{1.0,0.4,0.2}
\newcommand{\cob}{ \bf \color{MattOrange}}
\newcommand{\andd}{\ , \quad \text{and}  \quad}
\newcommand{\forr}{\ , \quad \text{for}  \quad}

\newcommand\citepls{{\bf\color{red}[[Cite Needed]]}}

\newcommand{\afour}{\ensuremath{A_4^{\text{tree}}}}
\definecolor{NUpurple}{RGB}{078,042,132}

\def\dj#1{{\color{NUpurple}\it \bf JJ: #1}} 



\author[1]{Alex Edison,}
\author[1]{James Mangan,}
\author[1]{Nicolas H. Pavao}

\affiliation[1]{Department of Physics and Astronomy, Northwestern
  University, Evanston, Illinois 60208, USA}

  
\title{Towards Globally Color-Dual All-Loop Integrands}

\abstract{ For now, let's put the things we need to check/compute here in the abstract. 
\begin{enumerate}
\item Does our construction manifest all the cuts for the two-loop n-gon?
\item What is the physics story here? Perturbative corrections to special Galilon, DBI for inflation? 
\item How can bubble on external legs that are zero by IBP inform generalizations beyond pions?
\end{enumerate}
}

  \newpage


\begin{document}
\maketitle
\flushbottom
 
\setstackgap{S}{6pt}
\setstackgap{L}{7pt}

\section{Introduction}

\label{sec:intro} Some refs. \cite{BCJreview, Bern:2022wqg, Adamo:2022dcm,Bern:2012uf, Neq44np, GravityFour,Bjerrum-Bohr:2013iza,Edison:2020uzf,Edison:2022jln,Cheung2016prv,He:2017spx,OneTwoLoopPureYMBCJ, Mogull:2015adi, Bern:2015ooa, Geyer:2019hnn,Johansson:2017bfl,Chen:2019ywi,Chen:2021chy,Brandhuber:2021bsf,Cheung:2021zvb,Ben-Shahar:2021zww,Cheung:2022mix, Ben-Shahar:2022ixa,Cachazo:2014xea,Carrasco:2016ldy,Elvang:2020kuj,Bern:2007xj,Carrasco:2013ypa,Craig:2019zkf,Monteiro:2022nqt,Bern:2017tuc,Bern:2017rjw,Bern:2019isl,Goroff:1985sz,Bern:2013uka,Alishahiha:2004eh,Creminelli:2005hu,Fergusson:2008ra,Carrasco:2015pla,Carrasco:2015rva,Carrasco:2015uma,BICEP:2021xfz,Kallosh:2021mnu,Green:1982sw,Mafra:2016mcc,Broedel:2013tta,Carrasco:2016ygv,Azevedo:2018dgo,Anastasiou:2004vj,vonManteuffel:2012np,Smirnov:2014hma,vonManteuffel:2014ixa,Smirnov:2019qkx,Smirnov:2020quc,Usovitsch:2020jrk,Maierhofer:2018gpa,Carrasco:2019yyn,Carrasco:2021ptp,Chi:2021mio,Bonnefoy:2021qgu,Carrasco:2022lbm,Carrasco:2022sck,Pavao:2022kog,Chen:2022shl,Chen:2023dcx,Brown:2023srz,Carrasco:2022jxn,Caron-Huot:2016icg, Chiodaroli:2021eug,Cangemi:2022abk,Cangemi:2022bew,Geiser:2022exp,Cheung:2022mkw,Fonseca:2019yya,Hays:2018zze,Alioli:2022fng}

\section{Review}
The goal is this paper is to compute $D$-dimensional color-dual numerators beyond one-loop. To this end compute two-loop color-dual numerators for NLSM that globally satisfy the duality between color and kinematics and elucidate the color-dual structure expected of Yang-Mills at all-loop order. To set the stage, it is worth distinguishing between the algebraic properties of functionally symmetric kinematics that satisfy color-kinematics globally, and some of the other constructions that appear in the literature. 

In this work, we will consider three different types of cubic representations, that we define below: 
\be
\stackrel{\Acubic{hgrey0}{}{nhpRed}{}{nhpRed}{fermion2}{}}{\text{Fundamental}}\qquad\qquad \stackrel{\cubic{hgrey0}{fermion2}{}{fermion2}{}{fermion2}{}}{\text{Bifundamental}}\qquad\qquad \stackrel{\Acubic{hgrey0}{}{nhpBlue}{}{nhpBlue}{}{nhpBlue}}{\text{Adjoint}}
\ee
In this review, we will present examples of this distinct kinematic representations that have appeared in the literature. Though from the get go, we stress that in this work we are interested in identifying the analog of adjoint representations of the kinematic numerators; those for which all legs are equally preferrenced by the functional dressing. The only non-trivial $S$-matrix in the literature that exhibits this property to all-loop order, is that of Zakharov-Mikhailov theory \cite{}. We will briefly review the perturbative construction of this theory below. 
\subsection{Self-dual Yang-Mills}
\be
\mathcal{L}^{\text{YM}} = -\frac{1}{4}F_{\mu\nu}F^{\mu\nu}
\ee
To obtain the Lagrangian for self-dual Yang-Mills, we applyy the self-dual condition that equates the field strength to its dual 2-form:
\be\label{SDcon}
F_{\mu\nu} = \frac{1}{2} \epsilon_{\mu\nu\rho\sigma}F^{\rho\sigma}
\ee
where $F_{\mu\nu} = \partial_\mu A_\nu -\partial_\nu A_\mu + g[A_\mu ,A_\nu]$ is the non-abelian $SU(N)$ field strength. Transforming from Cartesian coordinates to light cone coordinantes, $(t,x,y,z) \rightarrow (u,v,w,\bar{w})$, which take the following definition:
\be
u = t+x,\quad v=t-x, \quad w = y+iz
\ee
which gives the following expression for the invariant line element:
\be
ds^2 = 2(dudv - dwd\bar{w}) = dt^2 - d\vec{x}\cdot d\vec{x}
\ee
Applying this coordinate transformation to the self-dual condition of \eqn{SDcon}, in light-cone gauge, $A_u=0$, we obtain the follow set of equations that are one-to-one with the 4D self-dual condition:
\be
A_w = 0\,, \qquad \partial_u A_v = \partial_w A_{\bar{w}}\,.
\ee
The second equation implies the following that $A_v$ and $A_{\bar{w}}$ depend on a single degree of freedom, which we call $\Psi$. This constraint is immediately satisfied with the following definition of the gauge field:
\be
A_v = \frac{1}{2} \partial_w \Psi, \qquad A_{\bar{w}} = \frac{1}{2} \partial_u \Psi,
\ee
Expanding out the equations of motion for the Yang-Mills Lagrangian, yields the following expression in terms of the gauge fields:
\be
D^\mu F^{\mu\nu} =0\qquad  \Rightarrow\qquad \Box \Psi + ig [\partial_u \Psi, \partial_w \Psi ]=0
\ee
Adding back in the anti-holomorphic field $\bar{\Psi}$ as a Lagrange multiplier, we obtain the following interaction Lagrangian for self-dual Yang-Mills theory:
\be
\mathcal{L}^{\text{SDYM}} = \frac{1}{2}(\partial \bar{\Psi})(\partial \Psi) -i g \bar{\Psi} [\partial_u \Psi, \partial_w \Psi ]
\ee
\subsection{ZM-theory} The Lagrangian for Zakharov-Mikhailov (ZM) theory is a simple $SU(N)$ scalar theory defined in two spacetime dimensions:
\be
\mathcal{L}^{\text{ZM}} = \frac{1}{2}(\partial \varphi)^2 + g f^{abc} \varphi^a (\partial^\mu \varphi^b)( \tilde{\partial}_\mu \varphi^c) 
\ee
where $\tilde{\partial}_\mu = \epsilon_{\mu\nu}\partial^\nu $ and $\epsilon_{\mu\nu}$ is the two-dimensional Levi-Civita symbol. Expanded out into spacetime components, the momentum space Feynman rule for this theory is simply:
\be
\langle ab\rangle \equiv p_a^0 p_b^1 - p_a^1 p_b^0 = -\langle ba \rangle 
\ee
 From the interaction part of the Lagrangian, we can see that the three-point vertex is (1) functionally symmetric and (2) the kinematics are completely anti-symmetric: 
\be
\cubic{hgrey0}{}{nhpBlue}{}{nhpBlue}{}{nhpBlue} = \langle 12\rangle = \langle 23\rangle = \langle 31\rangle = -\langle 21\rangle
\ee
This important property is due to momentum conservation alone and does not need any additional kinematic constraints to satisfy the duality. In addition to anti-symmetry, the cubic graphs will satisfy jacobi off-shell on any internal edge due to the delta function identity of 2D Levi-Civita symbols:
\be
\langle ab \rangle \langle cd\rangle +\langle ac \rangle \langle db\rangle +\langle ad \rangle \langle bc\rangle = \delta^{[a}_c\delta^{b]}_d + \delta^{[a}_d\delta^{c]}_b+ \delta^{[a}_b\delta^{d]}_c=0
\ee
Theories with this property are the creme de la creme of color-dual gauge theories as they trivially satisfy color-kinematic algebraic relations off-shell. This allows one to trivially promote the duality from tree-level to all-loop order by dressing each cubic vertex with its own kinematic structure constant. 

Later in this section we will show that the interaction vertex is precisely that belonging to self-dual Yang-Mills (SDYM) theory in light-cone gauge. However, as we'll see, the kinetic term for these two theories differ -- the SDYM kinetic term transmutes between the holomorphic field $\Phi$, and the anti-holomorphic field, $\bar{\Phi}$, while the ZM scalar leaves the kinematic sector untouched. this is a critical distinction between the two theories that has important consequences for their color-dual constructions at loop-level.

While the color-dual nature of ZM theory is rather elegant, the loop-level construction introduces complications when applying dimensional regularization. Similar to the difficulties of renormalizing chiral fermions \cite{ 'tHooft}, there is ambiguity in promoting the integrands to formal $D$-dimensional expressions. 
\subsection{YZ-Theory}
\be
\mathcal{L}^{\text{YZ}} =\frac{1}{2} (\partial Y)^2 + (\partial Z)(\partial \bar{Z}) - g f^{abc} \left( \bar{Z}_{\mu\nu}Z^{\mu} Z^\nu + [Y,\partial_\mu Y] Z^\mu \right)
\ee
\be
\Acubic{hgrey0}{}{nhpRed}{}{nhpRed}{fermion2}{} = i \varepsilon_3 (p_1-p_2)\qquad \cubic{hgrey0}{fermion2}{}{fermion2}{}{fermion2}{} =i(\varepsilon_1 p_{2})(\varepsilon_2\bar{\varepsilon}_3) - i(\varepsilon_2 p_{1})(\varepsilon_1\bar{\varepsilon}_3)
\ee
We begin with a review of the construction of NLSM numerators at tree-level. For this work we extend the tree-level construction of $Y\!Z$ model of Cheung and Hsien \cite{Cheung2016prv}. In this construction, the generators of the kinematic algebra can be expressed as follows
\be\label{eq:FeynmanRuleYYZ}
T^a_{ij}= i \varepsilon_a(p_i-p_j)
\ee
where momentum conservation requires
\be
p_a + p_i + p_j =0
\ee
The kinematic half-ladder diagrams then takes on the following concise definition:
\be
n^{\text{NLSM}}_{(i|a_1a_2...a_n|j)} = (T^{a_1}T^{a_2}\cdots T^{a_n})_{ij}
\ee
Since there are on pole cancelling factors of $s_{ij} = (p_i+p_j)^2$, this definition of the chiral current algebra is manifestly cubic. Thus, the kinematic structure constants defined in terms of these generators are invariant under generalized gauge freedom. They can be defined implicitly below:
\be
[T^a,T^b]_{ij}= F^{a}_{\,b|c}T^c_{ij}
\ee
Given this definition, the Feynman rule associated with kinematic structure constant is simply:
\be\label{eq:FeynmanRuleXZZ}
i F^{a}_{\,b|c} = (\varepsilon_b p_{ab})(\varepsilon_a\bar{\varepsilon}_c) - (\varepsilon_a p_{ab})(\varepsilon_b\bar{\varepsilon}_c) 
\ee
where $p_{ab}=p_a+p_b$ and $\epsilon$ and $\bar{\epsilon}$ are the polarizations of the $Z$-vectors particle and its conjugate field, respectively. The on-shell state sum for the polarization vectors is simply:
\be
\sum_{\text{states}} \varepsilon^{\,\mu}_{(p)}\bar{\varepsilon}^{\,\nu}_{(-p)} = \eta^{\mu\nu}
\ee
Notice that the vector state sum is gauge fixed since the $Y\,Z$ model explicitly chooses Lorenz gauge for the $Z$ particles, $\partial_\mu Z^\mu=0$. To recover NLSM amplitudes from this these kinematic structure constants at tree-level, we simply plug in the following on-shell polarizations for the $Z$ and $\bar{Z}$ particles:
\be\label{eq:onShellZStates}
\varepsilon^\mu_{(p)} = p^\mu \qquad \bar{\varepsilon}^\mu_{(p)} = \frac{q^\mu}{pq}
\ee
where $q^2=0$ is some null reference momentum. With this, we can define tree-level pion scattering in two equivalent ways:
\be
A^{\text{NLSM}} = A(...,Y,...,Y,...) = A(...,\bar{Z},...) 
\ee
where the ellipses denote additional on-shell $Z$-particles. Subject to a particular gauge choice, the kinematic numerators in latter definition for pion scattering is equivalent to those $J$-theory, first written down by Cheung and one of the authors \cite{Cheung:2021zvb}. In this paradigm, the on-shell $\bar{Z}$ state is equivalent to the root leg appearing in the color-dual $J$-theory numerators. 

It is instructive to see how both of these constructions produce valid tree-level amplitudes for the pion. First we'll start with two $Y$ particles on legs 1 and 4. Applying the Feynman rules above, and plugging in on-shell states for the $Z$-particles produces the following $s$- and $t$-channel numerators:
\begin{align}
n^{YY}_s &= (T^2T^3)_{14} = s_{12}^2 
\\
 n^{YY}_t &=  F^{3}_{\,2|X}T^X_{14}  = s_{14}(s_{13}-s_{12})
\end{align}
Plugging these numerators into the ordered amplitudes $A(s,t)$ yields the desired result:
\be\label{eq:NLSMYZ4point}
A^{YY}_{(s,t)} = \frac{n^{YY}_s}{s_{12}}+\frac{ n^{YY}_t }{s_{14}} = s_{13}
\ee
Similarly we can do the same for the $Z$ and $\bar{Z}$ configuration. Below we have plugged in the on-shell $Z$-particle states, but have left the $\bar{Z}$ index free. This produces the following numerators:
\begin{align}
n^{\bar{Z}Z}_s &= {}_4(F^{3}F^{2})_{1} =  s_{12}(s_{14}-s_{13})p_2^{\mu_1}+s_{12}^2(p_3-p_4)^{\mu_1}
\\
 n^{\bar{Z}Z}_t &=   {}_2(F^{3}F^{4})_{1}  = s_{14}(s_{12}-s_{13})p_4^{\mu_1}+s_{14}^2(p_3-p_2)^{\mu_1}
\end{align}
where we have defined the short hand notation:
\be
{}_x(F^{a_1}F^{a_2}\cdots F^{a_n})_{y} \equiv F^{a_1}_{\,x|b_2}F^{a_2}_{\,b_2|b_3}\cdots F^{a_n}_{\,b_n|y}
\ee
Similarly, we find the following ordered amplitude when plugging in these numerators:
\be
A^{\bar{Z}Z}_{(s,t)} = \frac{n^{\bar{Z}Z}_s}{s_{12}}+\frac{ n^{\bar{Z}Z}_t }{s_{14}} = -s_{13}(p_2+p_3+p_4)^{\mu_1} = s_{13} \,p_1^{\mu_1}
\ee
Plugging in the on-shell polarizatoin of the conjugate field in \eqn{eq:onShellZStates} produces precisely the desired result of \eqn{eq:NLSMYZ4point}. Indeed, this construction is valid to all orders at tree-level. There are only two possible factorization channels the contribute the each of these amplitudes, the $YY$ cut and the $\bar{Z}Z$ cut :
\begin{align}
A(...,Y,...,Y,...) &\rightarrow A(...,Y,...,Y)A(Y,...,Y,...)
\\
&\rightarrow A(...,Y,...,Y,...,Z)A(\bar{Z},...)
\end{align}
\subsection{J-theory}
\be
\mathcal{L}^{\text{J}} = \frac{1}{2}(\partial J)(\partial \bar{J}) +g f^{abc} \bar{J}_{\mu\nu}J^\mu J^\nu
\ee
where as above, $\bar{J}_{\mu\nu} = \partial_\mu J_\nu - \partial_\nu J_\mu$. Subject to the Lorenz gauge condition, $\partial_\mu J^\mu =0$, this Lagrangian produces precisely the EOM for J-theory needed for Berends-Giele recursion at tree-level: 
\be
\partial^2 J^\mu - gf^{abc} J^\nu \partial_\nu J^\mu = 0 
\ee
As such, the single vertex in this theory is the same as for YZ theory: 
\be
\cubic{hgrey0}{fermion2}{}{fermion2}{}{fermion2}{} =i(\varepsilon_1 p_{2})(\varepsilon_2\bar{\varepsilon}_3) - i(\varepsilon_2 p_{1})(\varepsilon_1\bar{\varepsilon}_3)
\ee

\be
\mathcal{L}^{\text{SDYM}} = \frac{1}{2}(\partial \bar{\Phi} )^a(\partial  \Phi)^a +g f^{abc} \bar{\Phi} ^a (\partial_u \Phi)^b (\partial_v \Phi)^c
\ee
\be
(\varepsilon^+_1 k_2) - (\varepsilon^+_2 k_1) = [12]\left(\frac{\langle 2 q\rangle}{\langle 1 q\rangle} + \frac{\langle 1 q\rangle}{\langle 2 q\rangle} \right) \sim [12]
\ee
Now we are prepared to discuss the one-loop case. 

\section{One-loop}

\subsection{NLSM -- Nic}

Note from James:  I've explicitly checked the nice form of the 1-loop numerators at 4pt, 6pt, and 8pt.
\be
\simplebox = (2 f_\pi^{-1})^4 (l_1 k_1)(l_2 k_2)(l_3 k_3)(l_4 k_4)
\ee
where we define sequential loop momenta on the $n$-gon as $l_{i+1}= l_i+k_i$. Given this, the $n$-gon permits a equivalent expression in terms of inverse propagators:
\be
\simplebox =f_\pi^{-4} [12][23][34][41]
\ee
Where we have define the antisymmetric kinematic variable, $[LR] = l_i^L - l_R^2$
\subsection{BELs and integration zeros -- Nic}

At one-loop, weight counting tells us that the $n$-gon master numerator must have $n$ on-shell $Z$-particles. Unitarity requires that there are three distinct contributions from $Y\!Z$ theory -- the first from an off-shell $Y$-loop particle and then two more from different orientations of a $\bar{Z}Z$-loop.  Thus, $Y\!Z$ theory gives us the following one-loop $n$-gon numerator:
\be
N^{n\text{-gon},YZ}_{(12...n)} = (T^{1}T^{2}\cdots T^{n})+2\, (F^1F^2\cdots F^n)
\ee
where $(\,\cdots)$ indicates an internal contraction over the $YY$ and $\bar{Z}Z$ loops. Plugging in the Feynman rule of \eqn{eq:FeynmanRuleYYZ} and \eqn{eq:FeynmanRuleXZZ}, we can readily obtain expressions in terms of the internal loop factors $l_i$ and the external momenta $k_i$ (we define the $l_i$ loop momentum as that flowing into $k_i$ and out of $k_{i-1}$):
\be
N_{(12...n)}^{\text{NLSM}}=(T^{1}T^{2}\cdots T^{n}) = (l_1 k_1)(l_2 k_2) \cdots (l_n k_n)
\ee
and similarly so for the internal vector contribution:
\be
 (F^1F^2\cdots F^n) = (D-4)(l_1 k_1)(l_2 k_2) \cdots (l_n k_n) + \mathcal{O}(D^0)
\ee
The dimension dependent factor essentially counts that number of internal vector states. While this $n$-gon is manifestly color-dual, it does not produce the right cuts for NLSM. However, the scalar contribution, that comes dressed with an overall factor $(D-4)$ \textit{does} manifest the duality globally, and satisfies all the desired pion cuts. In order for the Feynman rules for $Y\!Z$ theory compute one-loop color-dual numerators consistent with NLSM cuts, we must add some additional states to cancel off he spurious poles, while preserving color-kinematics duality. We leave this as a direction of future work. 

While the $\bar{Z}Z$ vector loop spoils color-kinematics off-shell, the $Y\!Z$ loop alone gives us a desired expression for the $n$-gon. The important takeaway is that we now have a guess for the form of the off-shell three-point vertex that has a chance of manifesting the duality off-shell. The $n$-gon numerator above has scalar insertions of the following kinematic vertex:
\be
T^{a}_{LR} = k_a(l_L-l_{R}) = (l_L+l_{R}) (l_L-l_{R})  = l_L^2-l_{R}^2 
\ee
Given this structure, in the next section we will attempt to construct two-loop basis diagrams from these cubic vertex assignments and try to reverse engineer the particle content that produces these master numerators.

Before proceeding, we note a strange property of the $n$-gon numerator for the pions. In this form, Jacobi relations produce \textit{non-vanishing} values for bubbles on external legs (BELs). However, it is easy to see that the contribution integrates to zero after applying IBP relations/tensor reduction on the bubble. The BEL diagram can be reconstructed from the $n$-gon as follows:
\be
N^{\text{BEL}}_{1|2,34} = N^{\text{NLSM}}_{(1234,l)}-N^{\text{NLSM}}_{(1243,l)}-N^{\text{NLSM}}_{(1342,l)}+N^{\text{NLSM}}_{(1432,l)}
\ee
where we define the loop momentum to be in between the left most and right most leg on the box. Plugging in particular values for $l_i$, we obtain the following expression for the BEL
\be
N^{\text{BEL}}_{1|2,34} = s_{12} (l+k_1)^2 l^{\mu} k_1^{\nu} k_2^{[\mu} k_{[34]}^{\nu]} 
\ee
Notice there is an overall factor that cancels one of the propagators. Plugging this in, produces an integral of the following form
\be
\mathcal{I}^{\text{BEL}}_{1|2,34} = s_{12} k_1^{\nu} k_2^{[\mu} k_{[34]}^{\nu]} \int \frac{d^D l}{i\pi^{D/2}} \frac{l^\mu }{l^2-\mu^2} \sim   s_{12}(s_{13}-s_{14}) (\mu^2)^{D/2}
\ee
where we have introduced a mass regulator that will be proportional the the on-shell momentum inside the BEL, $\mu^2 \equiv k_1^2$. Thus, in large enough dimension, this integral suppresses the $\mu^{-2}$ divergence appearing in the denominator of the BEL diagram. 

\section{Two-loop}

Notes to myself: 
\begin{itemize}
\item ZM and 2D NLSM (PCM) are the same classically but not quantum mechanically.
This observation is due to other people but an easy way to see it.
ZM and 2D NLSM have the same EOM after a field redefinition.
This means that the classical physics is the same but of course the Lagrangians aren't the same and the field redefinition can change the path integral measure.
\item Forcing pions to match ZM in 2D does funny things.
The pion coupling constant is dropped but is implicitly set by the ordered 4pt amplitude which is normalized to be $-k_1 \cdot k_3$ in the package.
I've tabulated what happens when you take a general D NLSM answer and force it to match ZM in 2D.
Let's say that $n_\text{NLSM} = z n_\text{ZM}$ in 2D where I'm matching basis/master numerators.
For 4pt tree, $z=2/3$ (I matched the half ladder).
For 4pt 1-loop, $z$ is free/pure generalized gauge (I matched the box).
For 4pt 2-loop (when you impose certain $\ell$ power counting) $z$ is $216/565=\tfrac{2^3 3^3}{5 \cdot 113}$ (I matched the double box and penta-triangle).
For 6pt tree, $z$ is 12/35 (I matched the half ladder).
So you can match kinematic structure (the combination of Mandelstams) of NLSM to ZM in 2D but you can't get the coupling constant right at loop level, that is, ZM is a different theory quantum mechanically.
Actually these numbers make it look like you can't even match the theories classically but I guess for the theories to match classically you only need the amplitudes to agree and I matched numerators.
\item At 4pt 1-loop $\prod p_L^2 -p_R^2$ agrees with the ZM answer $\prod p_L \eps p_R$ up to some number.
At 6pt 1-loop this is not true and the two answers differ by alternating signs depending on the kinematic configuration.
\item You cannot lift ZM to general D NLSM in the democratic way that Nic described because it fails at 6pt tree.
\item To clean up the NLSM 4pt 2-loop answer I forced the answer to match ZM in 2D.
The ZM answer also only has certain powers of loop momenta $\ell^5$...$\ell^8$ where $\ell$ represents any combination of $\ell_1$ and $\ell_2$.
I enforce this $\ell^5$...$\ell^8$ behavior on the NLSM ansatz too.
\end{itemize}

\subsection{Cut construction, graph sym, Jacobi -- James}

Color-dual integrands are typically generated by constructing an ansatz and then imposing various constraints.
Color-kinematics duality requires writing the integrand in terms of cubic graphs.
Any theory can be written this way by artificially inserting ratios of propagators (can blow up YM or NLSM).
Obviously the ansatz must have the correct power counting and external states.
We will take NLSM as our example theory but later we will discuss YM.
Since pions are scalars the numerators will not carry any spinors or polarization vectors.
Every vertex in NLSM scales as $k^2$ so, for example, the numerator of a cubic 2-loop diagram scales as $k^12$.
There are three conditions that are imposed on the ansatz:  (1) the integrand must have the correct unitarity cuts, (2) numerators must respect graph symmetries, and (3) Jacobi relations amongst graphs must be satisfied.
The integrand must be written in terms of cubic graphs in order to apply the Jacobi relations.
This means that there are naively X cubic 4pt 2-loop diagrams.
However, many of these diagrams are related by Jacobi identities.
Give an example.
Through repeated application of the Jacobi identities, all of the cubic diagrams can be written in terms of the double box and penta-triangle.
The basis choice is not unique.
For example, any pair of the double box, penta-triangle and crossed box would work.
So you have to have two numerator ansatze, one for the double box and one for the crossed box.
What we are describing here is global color-kinematics duality.
It is possible to give each cubic graph its own numerator and then only impose Jacobi on the cuts (see ...).
It is not always possible to achieve global CK, but when it is doable, it is typically much easier than doing Jacobi on the cuts because the latter requires far more ansatz parameters.
Anyway, now the numerator of any diagram can be written in terms of the two basis numerators (double box and crossed box).
This is part of imposing Jacobi.
There are additional Jacobi relations that impose actual constraints on the ansatze, for example ASDF.
We allow BEL graphs to have nonzero numerators since they do not contribute to physical unitarity cuts.
Jacobi relations involving BEL graphs or tadpoles are simply not solved.

Requiring numerators to respect their graph symmetries is very convenient but is not completely necessary in order to perform the double copy.
At tree level, finding manifestly color dual numerators is often achieved at the expense of manifest Bose symmetry (see YZ theory, j-theory, Alex numerators from CHY, Henrik numerators).
Nonetheless, for a pure ansatz construction it is very reasonable to impose graph symmetries.
Note that you impose that the numerator of \emph{every} graph respects its graph automorphisms.
So the graph syms of the double box impose constraints, but, for example, the crossed box imposes constraints on the ansatz numerators too because the numerator of the crossed box can be written in terms of the two basis numerators.

Finally we have to impose unitarity.
We only do physical cuts, no aesthetic cuts.
The two most important cuts for pions (or any EMU) are ...
However there are other cuts that must be imposed as well.

References for Jacobi, unitarity, etc.

After you've imposed all of the essential stuff you get X free parameters...
There is lots of generalized gauge freedom, clearly there's room to impose extra constraints.
For N=4 the extra conditions are things like no triangles, strip off a factor of stAtree (which fails at 5-loops), manifest power counting, etc.
The obvious choice here is to match to ZM
There's one known theory that has manifest color-kinematics, manifest Bose sym, and NLSM power counting and that's ZM theory.

\subsection{Pions -- James}

\subsection{Matching to ZM -- James}

\subsection{Double-copy -- Alex }
\subsection{YM revisit -- Alex}

\section{Discussion}\label{sec:Discussion}


\paragraph{Acknowledgments} The authors would like to thank ... for insightful conversations, related collaboration, and encouragement throughout the completion of this work. This work was supported by the DOE under contract DE-SC0015910 and by the Alfred P. Sloan Foundation. N.H.P. additionally acknowledges the Northwestern University Amplitudes and Insight group, the Department of Physics and Astronomy, and Weinberg College for their generous support. 

\bibliographystyle{JHEP}
\bibliography{Refs_2loopNLSM}
\end{document}
