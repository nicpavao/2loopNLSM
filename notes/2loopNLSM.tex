\documentclass[12pt,letter]{article}
\usepackage{jheppub}
\usepackage{amsmath,amssymb}
\usepackage[dvipsnames]{xcolor}
%\usepackage{hyperref}
\usepackage{xspace}
\usepackage{ifdraft}
\usepackage{epstopdf}
\usepackage{slashed}
\usepackage{diagbox}
\usepackage{colortbl}
\usepackage{bbold}
\usepackage{tabularx}
\usepackage{graphicx}
\usepackage{stackengine}
\def\RA{\rlap{\scalebox{1.6}{$\leftrightarrow$}}}
\def\DA{\smash{\bclap{\scalebox{1.6}{$\downarrow$}}}}
\def\mystrut{\rule[-2ex]{0ex}{6ex}}

\usepackage{tikz-feynman} 
\usepackage{xcolor}
\hypersetup{
    colorlinks,
    linkcolor={hred},
    citecolor={hgreen},
    urlcolor={hblue}
}
\tikzset{graviton/.style={decorate, decoration={snake, amplitude=.4mm, segment length=1.5mm, pre length=.5mm, post length=.5mm}, double}}
\tikzfeynmanset{compat=1.1.0}
\immediate\write18{mkdir -p images}  %% Create `pgf-img` directory
\usetikzlibrary{external}  
          %% Load the `external` library
\immediate\write18{mkdir -p images}  %% Create `pgf-img` directory
\tikzexternalize[                     %% Activate externalization
  prefix=images/,                    %% Avoid cluttering the directory
 system call={                       %% Use lualatex in system call
    pdflatex \tikzexternalcheckshellescape -halt-on-error -shell-escape -interaction=batchmode -jobname="\image" "\texsource" || rm "\image.pdf"
  },
]




\definecolor{jaxoblue}{HTML}{0086FF}

\newcommand{\cP}{\rm{cov.}\pi}

\newcommand{\ie}{i.e.~}
\newcommand{\eg}{e.g.~}
\usepackage{xspace}
\newcommand{\Poincare}{Poincar\'e\xspace}

\newcommand{\cC}{g} % -- Crossing Count symbol.


\newcommand{\nkmc}[1][k]{N$^{#1}$MC\xspace}

\def\taua{{{\rm t}}}
\def\bartaua{{{\bar {\rm t}}}}


\definecolor{cutred}{RGB}{219,56,49}
\definecolor{hgreen}{RGB}{25,176,146}
\definecolor{hgreen1}{RGB}{175,230,175}
\definecolor{hblue}{RGB}{52,152,219}
\definecolor{hbluedark}{RGB}{36, 106, 160}
\definecolor{hblue1}{RGB}{255,255,166}
\definecolor{hred}{RGB}{216,83,117}
\definecolor{hreddark}{RGB}{151, 58, 81}
\definecolor{hred1}{RGB}{255,155,155}
\definecolor{cutred}{RGB}{219,56,49}
\definecolor{hgrey4}{RGB}{75,75,75}
\definecolor{hgrey5}{RGB}{50,50,50}
\definecolor{hgrey3}{RGB}{100,100,100}
\definecolor{hgrey}{RGB}{125,125,125}
\definecolor{hgrey2}{RGB}{125,125,125}
\definecolor{hgrey1}{RGB}{150,150,150}
\definecolor{hgrey0}{RGB}{175,175,175}
\definecolor{darkgreen}{RGB}{59,126,108}


\newcommand{\simplebox}{ {
\begin{tikzpicture}[baseline=(current  bounding  box.center)]
\begin{feynman}
\vertex (a1) at (-1,1){2};
\vertex (a2) at (1,-1){4};
\vertex (a3) at (1,1){3};
\vertex (a4) at (-1,-1){1};
\vertex (mid3) at (.5,.5);
\vertex (mid4) at (.5,-.5);
\vertex (mid5) at (-.5,.5);
\vertex (mid6) at (-.5,-.5);
\diagram{
(a4) --[ultra thick,](mid6),
(a3) --[ultra thick,](mid3),
(a2) --[ultra thick,](mid4),
(a1) --[ultra thick,](mid5),
(mid3) --[ultra thick,](mid4),
(mid5) --[ultra thick,](mid6),
(mid5) --[ultra thick,](mid3),
(mid4) --[ultra thick,](mid6),
};
\end{feynman}
\end{tikzpicture}
}
}

\newcommand{\xBox}{ {
\begin{tikzpicture}[baseline=(current  bounding  box.center)]
\begin{feynman}
\vertex (a1) at (-1.5,1){2};
\vertex (a2) at (1.5,-1){4};
\vertex (a3) at (1.5,1){3};
\vertex (a4) at (-1.5,-1){1};
\vertex (mid1) at (0,-.5);
\vertex (mid2) at (0,.5);
\vertex (mid3) at (1,.5);
\vertex (mid4) at (1,-.5);
\vertex (mid5) at (-1,.5);
\vertex (mid6) at (-1,-.5);
\vertex (mid7) at (-.5,0) {};
\diagram{
(a4) --[ultra thick,](mid6),
(a3) --[ultra thick,](mid3),
(a2) --[ultra thick,](mid4),
(a1) --[ultra thick,](mid5),
(mid1) --[ultra thick,](mid6),
(mid3) --[ultra thick,](mid2),
(mid1) --[ultra thick,](mid4),
(mid5) --[ultra thick,](mid7),
(mid7) --[ultra thick,](mid1),
(mid2) --[ultra thick,](mid6),
(mid5) --[ultra thick,](mid2),
(mid3) --[ultra thick,](mid4),
};
\end{feynman}
\end{tikzpicture}
}
}


\newcommand{\dBox}[7]{ {
\begin{tikzpicture}[baseline=(current  bounding  box.center)]
\begin{feynman}
\vertex (a1) at (-1.5,1){2};
\vertex (a2) at (1.5,-1){4};
\vertex (a3) at (1.5,1){3};
\vertex (a4) at (-1.5,-1){1};
\vertex (mid1) at (0,-.5);
\vertex (mid2) at (0,.5);
\vertex (mid3) at (1,.5) ;
\vertex (mid4) at (1,-.5) ;
\vertex (mid5) at (-1,.5);
\vertex (mid6) at (-1,-.5) ;
\diagram{
(a4) --[ultra thick,](mid6),
(a3) --[ultra thick,](mid3),
(a2) --[ultra thick,](mid4),
(a1) --[ultra thick,](mid5),
(mid1) --[ultra thick,#1](mid2),
(mid3) --[ultra thick,#2](mid2),
(mid1) --[ultra thick,#3](mid6),
(mid1) --[ultra thick,#4](mid4),
(mid5) --[ultra thick,#5](mid2),
(mid5) --[ultra thick,#6](mid6),
(mid3) --[ultra thick,#7](mid4),
};
\end{feynman}
\end{tikzpicture}
}
}


\newcommand{\dBoxR}[7]{ {
\begin{tikzpicture}[baseline=(current  bounding  box.center)]
\begin{feynman}
\vertex (a1) at (-1.5,1){2};
\vertex (a2) at (1.5,-1){4};
\vertex (a3) at (1.5,1){3};
\vertex (a4) at (-1.5,-1){1};
\vertex (mid1) at (0,-.5);
\vertex (mid2) at (0,.5);
\vertex (mid3) at (1,.5) ;
\vertex (mid4) at (1,-.5) ;
\vertex (mid5) at (-1,.5);
\vertex (mid6) at (-1,-.5) ;
\diagram{
(a4) --[ultra thick,](mid6),
(a3) --[ultra thick,hred](mid3),
(a2) --[ultra thick,hred](mid4),
(a1) --[ultra thick,](mid5),
(mid1) --[ultra thick,#1](mid2),
(mid3) --[ultra thick,#2](mid2),
(mid1) --[ultra thick,#3](mid6),
(mid1) --[ultra thick,#4](mid4),
(mid5) --[ultra thick,#5](mid2),
(mid5) --[ultra thick,#6](mid6),
(mid3) --[ultra thick,#7](mid4),
};
\end{feynman}
\end{tikzpicture}
}
}

\newcommand{\dBoxL}[7]{ {
\begin{tikzpicture}[baseline=(current  bounding  box.center)]
\begin{feynman}
\vertex (a1) at (-1.5,1){2};
\vertex (a2) at (1.5,-1){4};
\vertex (a3) at (1.5,1){3};
\vertex (a4) at (-1.5,-1){1};
\vertex (mid1) at (0,-.5);
\vertex (mid2) at (0,.5);
\vertex (mid3) at (1,.5) ;
\vertex (mid4) at (1,-.5) ;
\vertex (mid5) at (-1,.5);
\vertex (mid6) at (-1,-.5) ;
\diagram{
(a4) --[ultra thick,hred](mid6),
(a3) --[ultra thick,](mid3),
(a2) --[ultra thick,](mid4),
(a1) --[ultra thick,hred](mid5),
(mid1) --[ultra thick,#1](mid2),
(mid3) --[ultra thick,#2](mid2),
(mid1) --[ultra thick,#3](mid6),
(mid1) --[ultra thick,#4](mid4),
(mid5) --[ultra thick,#5](mid2),
(mid5) --[ultra thick,#6](mid6),
(mid3) --[ultra thick,#7](mid4),
};
\end{feynman}
\end{tikzpicture}
}
}



\DeclareMathOperator{\cut}{\mathcal{C}}

 \def\draftnote#1{{\color{red}\it #1}} 
\iffalse \def\draftnote#1{{\color{red}\it}} \fi

\def\sect#1{section~\ref{#1}}
\def\Fig#1{fig.~{\ref{#1}}}
\def\Fig#1{Fig.~{\ref{#1}}}
\def\Figs#1#2{figs.~{\ref{#1}} and {\ref{#2}}}
\def\Figs#1#2{Figs.~{\ref{#1}} and {\ref{#2}}}

\def\tab#1{table~{\ref{#1}}}
\def\Tab#1{Table~{\ref{#1}}}
\def\tabs#1#2{tables~{\ref{#1}} and {\ref{#2}}}
\def\Tabs#1#2{TablesnFigs.~{\ref{#1}} and {\ref{#2}}}

\def\spa#1.#2{\left\langle#1\,#2\right\rangle}
\def\spb#1.#2{\left[#1\,#2\right]}
\def\spash#1.#2{\spa{\smash{#1}}.{\smash{#2}}}
\def\spbsh#1.#2{\spb{\smash{#1}}.{\smash{#2}}}
\def\sand#1.#2.#3{%
\left\langle\smash{#1}{\vphantom1}^{-}\right|{#2}%
\left|\smash{#3}{\vphantom1}^{-}\right\rangle}
\def\sandpp#1.#2.#3{%
\left\langle\smash{#1}{\vphantom1}^{+}\right|{#2}%
\left|\smash{#3}{\vphantom1}^{+}\right\rangle}
\def\sandpm#1.#2.#3{%
\left\langle\smash{#1}{\vphantom1}^{+}\right|{#2}%
\left|\smash{#3}{\vphantom1}^{-}\right\rangle}
\def\sandmp#1.#2.#3{%
\left\langle\smash{#1}{\vphantom1}^{-}\right|{#2}%
\left|\smash{#3}{\vphantom1}^{+}\right\rangle}
\def\Shift#1#2{{[#1,#2\rangle}}
\def\twoloop{{2 \mbox{-} \rm loop}}

\def\tree{{\rm tree}}
\def\pol{\varepsilon}
\def\Tr{\, {\rm Tr}}
\def\tr{\, {\rm tr}}
\def\eps{\varepsilon}
\def\e{\varepsilon}
\def\ep{\varepsilon}
\def\SYM{MSYM}
\def\nn{\nonumber}
\def\sec#1{section~\ref{#1}}
\def\eqn#1{eq.~(\ref{#1})}
\def\Eqn#1{Equation~(\ref{#1})}
\def\eqns#1#2{eqs.~(\ref{#1}) and~(\ref{#2})}
\def\Eqns#1#2{Eqs.~(\ref{#1}) and~(\ref{#2})}
\def\Figref#1{Fig.~\ref{#1}}
\def\secref#1{section~\ref{#1}}
\def\Neqfour{{{\cal N}=4}}
\def\NeqFour{{{\cal N}=4}}
\def\Neqeight{{{\cal N}=8}}
\def\NeqEight{{{\cal N}=8}}
\def\NeqOne{{{\cal N}=1}}
\def\Fact{{\cal F}}
\def\f{\widetilde f}
\def\be{\begin{equation}}
\def\ee{\end{equation}}
\def\bea{\begin{eqnarray}}
\def\eea{\end{eqnarray}}
\def\ba{\begin{eqnarray}}
\def\ea{\end{eqnarray}}
\def\Ksl{{\s K}}
\def\ksl{\s{k}}
\def\Perm{{\cal P}}
\def\M{{\cal M}}
\def\ve{\varepsilon}
\def\tlambda{{\tilde\lambda}}
\def\MHVbar{$\overline{\hbox{MHV}}$}
\def\NMHVbar{$\overline{\hbox{NMHV}}$}
\def\P{{\rm P}}
\def\NHP{{\rm NHP}}
\def\mud{\lambda}
\def\bowtie{{\rm bow\mbox{-}tie}}
\newcommand{\cred}{\bf \color{red}}
 \newcommand{\cblue}{\color{blue}}
 \definecolor{MattOrange}{rgb}{1.0,0.4,0.2}
\newcommand{\cob}{ \bf \color{MattOrange}}
\newcommand{\andd}{\ , \quad \text{and}  \quad}
\newcommand{\forr}{\ , \quad \text{for}  \quad}

\newcommand\citepls{{\bf\color{red}[[Cite Needed]]}}

\newcommand{\afour}{\ensuremath{A_4^{\text{tree}}}}
\definecolor{NUpurple}{RGB}{078,042,132}

\def\dj#1{{\color{NUpurple}\it \bf JJ: #1}} 



\author[1]{Alex Edison,}
\author[1]{James Mangan,}
\author[1]{Nicolas H. Pavao}

\affiliation[1]{Department of Physics and Astronomy, Northwestern
  University, Evanston, Illinois 60208, USA}

  
\title{Color-dual NLSM multi-loop integrands}

\abstract{ For now, let's put the things we need to check/compute here in the abstract. 
\begin{itemize}
\item (1) Does our construction manifest all the cuts for the two-loop n-gon?
\item  (2) Is there a natural extension to 3-loop 
\item (3)  Is there a simple Feynman rule/ghost we can add to correct the one-loop n-gon of YZ theory
\item (4) Can we make our two-loop numerators functionally symmetric 
\item (5) What is the physics story here? Perturbative corrections to special Galilon, DBI for inflation? 
\item (6) What is the rule for cuts we don't care about? Like the weird snaking diagrams that James identified?
\item (7) How can bubble on external legs that are zero by IBP inform generalizations beyond pions?
\end{itemize}
}

  \newpage


\begin{document}
\maketitle
\flushbottom
 
\setstackgap{S}{6pt}
\setstackgap{L}{7pt}

\section{Introduction}\label{sec:intro} Some refs. \cite{BCJreview, Bern:2022wqg, Adamo:2022dcm,Bern:2012uf, Neq44np, GravityFour,Bjerrum-Bohr:2013iza,Edison:2020uzf,Edison:2022jln,Cheung2016prv,He:2017spx,OneTwoLoopPureYMBCJ, Mogull:2015adi, Bern:2015ooa, Geyer:2019hnn,Johansson:2017bfl,Chen:2019ywi,Chen:2021chy,Brandhuber:2021bsf,Cheung:2021zvb,Ben-Shahar:2021zww,Cheung:2022mix, Ben-Shahar:2022ixa,Cachazo:2014xea,Carrasco:2016ldy,Elvang:2020kuj,Bern:2007xj,Carrasco:2013ypa,Craig:2019zkf,Monteiro:2022nqt,Bern:2017tuc,Bern:2017rjw,Bern:2019isl,Goroff:1985sz,Bern:2013uka,Alishahiha:2004eh,Creminelli:2005hu,Fergusson:2008ra,Carrasco:2015pla,Carrasco:2015rva,Carrasco:2015uma,BICEP:2021xfz,Kallosh:2021mnu,Green:1982sw,Mafra:2016mcc,Broedel:2013tta,Carrasco:2016ygv,Azevedo:2018dgo,Anastasiou:2004vj,vonManteuffel:2012np,Smirnov:2014hma,vonManteuffel:2014ixa,Smirnov:2019qkx,Smirnov:2020quc,Usovitsch:2020jrk,Maierhofer:2018gpa,Carrasco:2019yyn,Carrasco:2021ptp,Chi:2021mio,Bonnefoy:2021qgu,Carrasco:2022lbm,Carrasco:2022sck,Pavao:2022kog,Chen:2022shl,Chen:2023dcx,Brown:2023srz,Carrasco:2022jxn,Caron-Huot:2016icg, Chiodaroli:2021eug,Cangemi:2022abk,Cangemi:2022bew,Geiser:2022exp,Cheung:2022mkw,Fonseca:2019yya,Hays:2018zze,Alioli:2022fng}
\section{Tree-level}
We begin with a review of the construction of NLSM numerators at tree-level. For this work we extend the tree-level construction of $Y\!Z$ model of Cheung and Hsien \cite{Cheung2016prv}. In this construction, the generators of the kinematic algebra can be expressed as follows
\be\label{eq:FeynmanRuleYYZ}
T^a_{ij}= i \varepsilon_a(p_i-p_j)
\ee
where momentum conservation requires
\be
p_a + p_i + p_j =0
\ee
The kinematic half-ladder diagrams then takes on the following concise definition:
\be
n^{\text{NLSM}}_{(i|a_1a_2...a_n|j)} = (T^{a_1}T^{a_2}\cdots T^{a_n})_{ij}
\ee
Since there are on pole cancelling factors of $s_{ij} = (p_i+p_j)^2$, this definition of the chiral current algebra is manifestly cubic. Thus, the kinematic structure constants defined in terms of these generators are invariant under generalized gauge freedom. They can be defined implicitly below:
\be
[T^a,T^b]_{ij}= F^{a}_{\,b|c}T^c_{ij}
\ee
Given this definition, the Feynman rule associated with kinematic structure constant is simply:
\be\label{eq:FeynmanRuleXZZ}
i F^{a}_{\,b|c} = (\varepsilon_b p_{ab})(\varepsilon_a\bar{\varepsilon}_c) - (\varepsilon_a p_{ab})(\varepsilon_b\bar{\varepsilon}_c) 
\ee
where $p_{ab}=p_a+p_b$ and $\epsilon$ and $\bar{\epsilon}$ are the polarizations of the $Z$-vectors particle and its conjugate field, respectively. The on-shell state sum for the polarization vectors is simply:
\be
\sum_{\text{states}} \varepsilon^{\,\mu}_{(p)}\bar{\varepsilon}^{\,\nu}_{(-p)} = \eta^{\mu\nu}
\ee
Notice that the vector state sum is gauge fixed since the $Y\,Z$ model explicitly chooses Lorenz gauge for the $Z$ particles, $\partial_\mu Z^\mu=0$. To recover NLSM amplitudes from this these kinematic structure constants at tree-level, we simply plug in the following on-shell polarizations for the $Z$ and $\bar{Z}$ particles:
\be\label{eq:onShellZStates}
\varepsilon^\mu_{(p)} = p^\mu \qquad \bar{\varepsilon}^\mu_{(p)} = \frac{q^\mu}{pq}
\ee
where $q^2=0$ is some null reference momentum. With this, we can define tree-level pion scattering in two equivalent ways:
\be
A^{\text{NLSM}} = A(...,Y,...,Y,...) = A(...,\bar{Z},...) 
\ee
where the ellipses denote additional on-shell $Z$-particles. Subject to a particular gauge choice, the kinematic numerators in latter definition for pion scattering is equivalent to those $J$-theory, first written down by Cheung and one of the authors \cite{Cheung:2021zvb}. In this paradigm, the on-shell $\bar{Z}$ state is equivalent to the root leg appearing in the color-dual $J$-theory numerators. 

It is instructive to see how both of these constructions produce valid tree-level amplitudes for the pion. First we'll start with two $Y$ particles on legs 1 and 4. Applying the Feynman rules above, and plugging in on-shell states for the $Z$-particles produces the following $s$- and $t$-channel numerators:
\begin{align}
n^{YY}_s &= (T^2T^3)_{14} = s_{12}^2 
\\
 n^{YY}_t &=  F^{3}_{\,2|X}T^X_{14}  = s_{14}(s_{13}-s_{12})
\end{align}
Plugging these numerators into the ordered amplitudes $A(s,t)$ yields the desired result:
\be\label{eq:NLSMYZ4point}
A^{YY}_{(s,t)} = \frac{n^{YY}_s}{s_{12}}+\frac{ n^{YY}_t }{s_{14}} = s_{13}
\ee
Similarly we can do the same for the $Z$ and $\bar{Z}$ configuration. Below we have plugged in the on-shell $Z$-particle states, but have left the $\bar{Z}$ index free. This produces the following numerators:
\begin{align}
n^{\bar{Z}Z}_s &= {}_4(F^{3}F^{2})_{1} =  s_{12}(s_{14}-s_{13})p_2^{\mu_1}+s_{12}^2(p_3-p_4)^{\mu_1}
\\
 n^{\bar{Z}Z}_t &=   {}_2(F^{3}F^{4})_{1}  = s_{14}(s_{12}-s_{13})p_4^{\mu_1}+s_{14}^2(p_3-p_2)^{\mu_1}
\end{align}
where we have defined the short hand notation:
\be
{}_x(F^{a_1}F^{a_2}\cdots F^{a_n})_{y} \equiv F^{a_1}_{\,x|b_2}F^{a_2}_{\,b_2|b_3}\cdots F^{a_n}_{\,b_n|y}
\ee
Similarly, we find the following ordered amplitude when plugging in these numerators:
\be
A^{\bar{Z}Z}_{(s,t)} = \frac{n^{\bar{Z}Z}_s}{s_{12}}+\frac{ n^{\bar{Z}Z}_t }{s_{14}} = -s_{13}(p_2+p_3+p_4)^{\mu_1} = s_{13} \,p_1^{\mu_1}
\ee
Plugging in the on-shell polarizatoin of the conjugate field in \eqn{eq:onShellZStates} produces precisely the desired result of \eqn{eq:NLSMYZ4point}. Indeed, this construction is valid to all orders at tree-level. There are only two possible factorization channels the contribute the each of these amplitudes, the $YY$ cut and the $\bar{Z}Z$ cut :
\begin{align}
A(...,Y,...,Y,...) &\rightarrow A(...,Y,...,Y)A(Y,...,Y,...)
\\
&\rightarrow A(...,Y,...,Y,...,Z)A(\bar{Z},...)
\end{align}
Now we are prepared to discuss the one-loop case. 
\section{One-loop}
At one-loop, weight counting tells us that the $n$-gon master numerator must have $n$ on-shell $Z$-particles. Unitarity requires that there are three distinct contributions from $Y\!Z$ theory -- the first from an off-shell $Y$-loop particle and then two more from different orientations of a $\bar{Z}Z$-loop.  Thus, $Y\!Z$ theory gives us the following one-loop $n$-gon numerator:
\be
N^{n\text{-gon},YZ}_{(12...n)} = (T^{1}T^{2}\cdots T^{n})+2\, (F^1F^2\cdots F^n)
\ee
where $(\,\cdots)$ indicates an internal contraction over the $YY$ and $\bar{Z}Z$ loops. Plugging in the Feynman rule of \eqn{eq:FeynmanRuleYYZ} and \eqn{eq:FeynmanRuleXZZ}, we can readily obtain expressions in terms of the internal loop factors $l_i$ and the external momenta $k_i$ (we define the $l_i$ loop momentum as that flowing into $k_i$ and out of $k_{i-1}$):
\be
N_{(12...n)}^{\text{NLSM}}=(T^{1}T^{2}\cdots T^{n}) = (l_1 k_1)(l_2 k_2) \cdots (l_n k_n)
\ee
and similarly so for the internal vector contribution:
\be
 (F^1F^2\cdots F^n) = (D-4)(l_1 k_1)(l_2 k_2) \cdots (l_n k_n) + \mathcal{O}(D^0)
\ee
The dimension dependent factor essentially counts that number of internal vector states. While this $n$-gon is manifestly color-dual, it does not produce the right cuts for NLSM. However, the scalar contribution, that comes dressed with an overall factor $(D-4)$ \textit{does} manifest the duality globally, and satisfies all the desired pion cuts. In order for the Feynman rules for $Y\!Z$ theory compute one-loop color-dual numerators consistent with NLSM cuts, we must add some additional states to cancel off he spurious poles, while preserving color-kinematics duality. We leave this as a direction of future work. 

While the $\bar{Z}Z$ vector loop spoils color-kinematics off-shell, the $Y\!Z$ loop alone gives us a desired expression for the $n$-gon. The important takeaway is that we now have a guess for the form of the off-shell three-point vertex that has a chance of manifesting the duality off-shell. The $n$-gon numerator above has scalar insertions of the following kinematic vertex:
\be
T^{a}_{LR} = k_a(l_L-l_{R}) = (l_L+l_{R}) (l_L-l_{R})  = l_L^2-l_{R}^2 
\ee
Given this structure, in the next section we will attempt to construct two-loop basis diagrams from these cubic vertex assignments and try to reverse engineer the particle content that produces these master numerators.

Before proceeding, we note a strange property of the $n$-gon numerator for the pions. In this form, Jacobi relations produce \textit{non-vanishing} values for bubbles on external legs (BELs). However, it is easy to see that the contribution integrates to zero after applying IBP relations/tensor reduction on the bubble. The BEL diagram can be reconstructed from the $n$-gon as follows:
\be
N^{\text{BEL}}_{1|2,34} = N^{\text{NLSM}}_{(1234,l)}-N^{\text{NLSM}}_{(1243,l)}-N^{\text{NLSM}}_{(1342,l)}+N^{\text{NLSM}}_{(1432,l)}
\ee
where we define the loop momentum to be in between the left most and right most leg on the box. Plugging in particular values for $l_i$, we obtain the following expression for the BEL
\be
N^{\text{BEL}}_{1|2,34} = s_{12} (l+k_1)^2 l^{\mu} k_1^{\nu} k_2^{[\mu} k_{[34]}^{\nu]} 
\ee
Notice there is an overall factor that cancels one of the propagators. Plugging this in, produces an integral of the following form
\be
\mathcal{I}^{\text{BEL}}_{1|2,34} = s_{12} k_1^{\nu} k_2^{[\mu} k_{[34]}^{\nu]} \int \frac{d^D l}{i\pi^{D/2}} \frac{l^\mu }{l^2-\mu^2} \sim   s_{12}(s_{13}-s_{14}) (\mu^2)^{D/2}
\ee
where we have introduced a mass regulator that will be proportional the the on-shell momentum inside the BEL, $\mu^2 \equiv k_1^2$. Thus, in large enough dimension, this integral suppresses the $\mu^{-2}$ divergence appearing in the denominator of the BEL diagram. 
\section{Two-loop}
Let's start with a two-loop four-point example. At this order, the basis diagrams for color-dual representations are any two of the Jacobi triple, the double box, the penta-triangle or the cross-box. First we introduce a bit of shorthand notion:
\be
[LR] \equiv  l_L^2 -l_R^2 \qquad [M] \equiv  l_M^2
\ee
Now let's start with the double-box. Below we treat the color legs as $Y$ particles that will get dressed with factors of $[LR]$. Thus, we wish to manifest the following cut:
\be
N^{\text{2box}}_M\equiv \dBox{}{hred}{hred}{hred}{hred}{hred}{hred} = [12][23][34][45][56][61]
\ee
In addition, we want the following cuts to be simultaneously satisfied:
\be
N^{\text{2box}}_R\equiv \dBoxR{hred}{}{hred}{}{hred}{hred}{hred} = [61][12][27][76][4]^2
\ee
\be
N^{\text{2box}}_L\equiv \dBoxL{hred}{hred}{}{hred}{}{hred}{hred}=[45][57][73][34][1]^2
\ee
Where the middle cut takes $l_7^2\rightarrow 0$, the right cut takes $l_3^2,l_5^2\rightarrow 0$ and the left cut takes $l_2^2,l_6^2\rightarrow 0$. Applying these cuts to all the diagrams above, we obtain the following expressions:
\be
\text{Cut}_R(N^{\text{2box}}_M) = \Delta_R + N^{\text{2box}}_R \qquad \text{Cut}_L(N^{\text{2box}}_M) = \Delta_L + N^{\text{2box}}_L
\ee
However, we find that the remainder terms $\Delta_{L/R}$ are proportional to $l_7^2$, and thus vanish on the middle cut:
\be
\text{Cut}_M(\Delta_{L/R})  = 0
\ee
Similarly, applying the complement left/right cuts to the right/left remainder terms, we obtain a single kinematic function:
\be
\text{Cut}_L(\Delta_R) = \text{Cut}_R(\Delta_L) = \Delta_{LR}
\ee
Thus, the double box numerator that simultaneously manifests all of the cut contributing to NLSM are as follows:
\be
\boxed{N^{\text{NLSM}}_{\text{2box}} = N^{\text{2box}}_M - (\Delta_{L}+\Delta_{R})+\Delta_{LR}}
\ee
Applying the cut relations above, we find that this double box numerator factorizes to the desired expressions:
\be
\text{Cut}_{X}(N^{\text{NLSM}}_{\text{2box}} )= N^{\text{2box}}_{X}
\ee
The kinematic expressions obtained for the remainder functions are provided below:
\begin{align}
\Delta_L &= \frac{1}{2}[61][12]([27]+[67])[4]^2[7] 
\\
 \Delta_R &= \frac{1}{2}[34][45]([37]+[57])[1]^2[7]
 \\
\Delta_{LR}&= [1]^2[4]^2[7]^2
\end{align}
While these numerators are not functionally automorphic like the one-loop $n$-gon numerators, they do obey color-kinematics when evaluated on-shell and thus should be valid representations for double copy construction. 

\section{Notes}\label{sec:Notes}

\begin{itemize}
\item ZM and 2D NLSM (PCM) are the same classically but not quantum mechanically.
This observation is due to other people but an easy way to see it.
ZM and 2D NLSM have the same EOM after a field redefinition.
This means that the classical physics is the same but of course the Lagrangians aren't the same and the field redefinition can change the path integral measure.
\item Forcing pions to match ZM in 2D does funny things.
The pion coupling constant is dropped but is implicitly set by the ordered 4pt amplitude which is normalized to be $-k_1 \cdot k_3$ in the package.
I've tabulated what happens when you take a general D NLSM answer and force it to match ZM in 2D.
Let's say that $n_\text{NLSM} = z n_\text{ZM}$ in 2D where I'm matching basis/master numerators.
For 4pt tree, $z=2/3$ (I matched the half ladder).
For 4pt 1-loop, $z$ is free/pure generalized gauge (I matched the box).
For 4pt 2-loop (when you impose certain $\ell$ power counting) $z$ is $216/565=\tfrac{2^3 3^3}{5 \cdot 113}$ (I matched the double box and penta-triangle).
For 6pt tree, $z$ is 12/35 (I matched the half ladder).
So you can match kinematic structure (the combination of Mandelstams) of NLSM to ZM in 2D but you can't get the coupling constant right at loop level, that is, ZM is a different theory quantum mechanically.
Actually these numbers make it look like you can't even match the theories classically but I guess for the theories to match classically you only need the amplitudes to agree and I matched numerators.
\item At 4pt 1-loop $\prod p_L^2 -p_R^2$ agrees with the ZM answer $\prod p_L \eps p_R$ up to some number.
At 6pt 1-loop this is not true and the two answers differ by alternating signs depending on the kinematic configuration.
\item You cannot lift ZM to general D NLSM in the democratic way that Nic described because it fails at 6pt tree.
\item To clean up the NLSM 4pt 2-loop answer I forced the answer to match ZM in 2D.
The ZM answer also only has certain powers of loop momenta $\ell^5$...$\ell^8$ where $\ell$ represents any combination of $\ell_1$ and $\ell_2$.
I enforce this $\ell^5$...$\ell^8$ behavior on the NLSM ansatz too.
\end{itemize}

\section{Discussion}\label{sec:Discussion}


\paragraph{Acknowledgments} The authors would like to thank ... for insightful conversations, related collaboration, and encouragement throughout the completion of this work. This work was supported by the DOE under contract DE-SC0015910 and by the Alfred P. Sloan Foundation. N.H.P. additionally acknowledges the Northwestern University Amplitudes and Insight group, the Department of Physics and Astronomy, and Weinberg College for their generous support. 

\bibliographystyle{JHEP}
\bibliography{Refs_2loopNLSM}
\end{document}
