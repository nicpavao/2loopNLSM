%\documentclass[notitlepage, 11pt]{revtex4-1}
%\linespread{1.4}

\documentclass[notitlepage, 12pt]{revtex4-1}
\linespread{1.2}

\usepackage{geometry}
\geometry{verbose,tmargin=1in,bmargin=1in,lmargin=0.75in,rmargin=0.75in}
\usepackage{color}
\usepackage{babel}
\usepackage{mathtools}
\usepackage{amsmath}
\usepackage{amssymb}
\usepackage{graphicx}
%\usepackage{esint}
\usepackage[T1]{fontenc}

\usepackage{enumitem}   


\makeatletter
%%%%%%%%%%%%%%%%%%%%%%%%%%%%%% User specified LaTeX commands.
\usepackage{url}
\usepackage{amsmath}
\usepackage{xcolor}
\usepackage{hyperref}
 \hypersetup{colorlinks=true,citecolor=blue,linkcolor=blue}
\renewcommand{\bibfont}{\footnotesize}
\usepackage{fancyheadings}
%\pagestyle{fancy}
\renewcommand{\headrulewidth}{12pt}

\makeatother

\usepackage{babel}
\begin{document}

\title{Response to Editor: JHEP\_031P\_1223}

\date{\today}

\maketitle
Dear Editor, 

\

We would like to begin by sincerely thanking the referee for their careful reading of the manuscript.
We believe their thoughtful comments and suggestions have produced a better paper.
The referee suggested more context for the two-loop results, and noted that a comparison with $\mathcal{N}=4$ super-Yang Mills (sYM) was absent. We agree that adding this context provides a necessary framing of the main technical results.
To this end, below is a summary of the modifications to the manuscript:
\begin{itemize}
\item A small comment has been added to the introduction at the top of page 2 reminding readers that color-dual representations of sYM have been found to high loop order.
\item At the bottom of page 2, when stating that there is no local 4pt 2-loop numerator for pure YM we remind readers that such a numerator exists for sYM.
\item At the end of the introduction at the top of page 3 we clarify that the bowtie cut failure is specific to pure YM.
We go on to speculate that this is because sYM and NLSM are simpler theories, at least as far as color-kinematics is concerned.
\item In addition to the comments in the introduction and section 4, we have included one additional modification.
Section 2.3 deals with ansatz locality so this seemed like a natural place to mention that the sYM ans\"atze incorporate locality as a key simplifying assumption.
At the bottom of page 7 and the top of page 8, we emphasize that the sYM numerators through four loops are local.
We comment in footnote 4 that the full set of local numerators has never been fully explored for sYM at 4pt 5-loop.
\item The first paragraph in section 4 closes by putting our results in the context of sYM, specifically by reminding readers of the state of the art in that theory. Another such reminder is included in footnote 10 at the beginning of section 4.3. 
\item After stating the bowtie failure condition for pure YM in (4.9) we give specific reasons for why NLSM and sYM are more privileged than pure YM consistent with the ``bow-tie'' cut. 
\end{itemize}


We would like to thank the referees again for their carefully considered feedback.
Now that the referees comments have been addressed, and we believe that the improved the readability of the manuscript makes it an excellent fit for JHEP.

\

Best regards,

Alex Edison, James Mangan, and Nic Pavao

\end{document}